\documentclass[12pt]{article}
\usepackage{amsmath,amssymb,amsmath}
\usepackage[spanish]{babel}
\usepackage{graphicx}
\usepackage{multicol}
\usepackage{hyperref}
\usepackage{enumerate}
\usepackage{esint}
\usepackage{tikz}
\usepackage{textcomp}
\pagestyle{empty}
\parindent=0pc
\setlength{\oddsidemargin}{0cm} \setlength{\evensidemargin}{0cm}
\setlength{\textwidth}{16cm} \setlength{\textheight}{24cm}
\setlength{\topmargin}{-3.0cm}
\usepackage{sagetex}
\newcommand{\Bold}[1]{\mathbb{#1}}

\newcounter{problems}

\begin{document}

\noindent
\textbf{Universidad de Concepción}\\
\textbf{Facultad de Ciencias Físicas y Matemáticas}\\
\textbf{Departamento de Matemática}
\begin{center}
{\bfseries Listado 09: Operaciones en campos finitos\\ Álgebra con Software (527282)}\\
\end{center}

\begin{enumerate}
    \item Sean $E,F$ campos con $E\subseteq F$. Sean $p(x)\in E[x]$ un polinomio irreducible con coeficientes en $E$ y $a\in F$ una raíz del polinomio $p$. Mostrar: si $a$ es raíz de un polinomio $q(x)\in E[x]$, entonces $q$ es múltiplo de $p$.
    
    \item Usar el ejercicio anterior para mostrar el \emph{teorema de las raíces conjugadas}: si $z\in\mathbb{C}\setminus\mathbb{R}$ es raíz de un polinomio real $p(x)\in\mathbb{R}[x]$, entonces el conjugado $\overline{z}$ también es raíz de $p$.
    
    \item Determinar, a mano y después con software, los determinantes de cada una de las siguientes matrices. En el caso de que la matriz sea invertible, determinar también su inversa.
    

    \begin{enumerate}
      \item
      \begin{sageblock}
        R = GF(5)
        M = matrix(R, [[1,3], [3,2]])
      \end{sageblock}
      
      $$M=\sage{M}$$ con coeficientes en el campo $R$: \sage{R.__repr__()}.
      
      \hrulefill
      
      \item
      \begin{sageblock}
        R = GF(7)
        M = matrix(R, [[1,3], [3,2]])
      \end{sageblock}
      
      $$M=\sage{M}$$ con coeficientes en el campo $R$: \sage{R.__repr__()}.
      
      \hrulefill
      
      \item
      \begin{sageblock}
        x = polygen(GF(5), "x")
        R.<a> = GF(5).extension(x^2+x+1, "a")
        M = matrix(R, [[1,a],[a,2*a+1]])
      \end{sageblock}
      
      $$M=\sage{M}$$ con coeficientes en el campo $R$: \sage{R.__repr__()} donde $a$ es raíz de $\sage{a.minpoly()}$.
      
      \hrulefill
      
      \item
      \begin{sageblock}
        x = polygen(GF(5), "x")
        R.<a> = GF(5).extension(x^3+x+1, "a")
        M = matrix(R, [[1,a],[a,2*a+1]])
      \end{sageblock}
      
      $$M=\sage{M}$$ con coeficientes en el campo $R$: \sage{R.__repr__()} donde $a$ es raíz de $\sage{a.minpoly()}$.
      
    \end{enumerate}
    
    \item Determinar, a mano y después con software, los conjuntos solución de cada uno de los siguientes sistemas lineales. Indicar además la cardinalidad de cada conjunto solución.
    \begin{enumerate}
      \item
      \begin{sageblock}
        R = GF(7)
        M = matrix(R, [[1,2,3],[2,6,2],[5,4,3]])
        b = column_matrix(R, [1,5,3])
        M_b = M.augment(b, subdivide=True)
      \end{sageblock}
      $$M_b = \sage{M_b}$$ con coeficientes en el campo $R$: \sage{R.__repr__()}.
      
      \hrulefill
      
      \item
      \begin{sageblock}
        R = GF(5)
        M = matrix(R, [[1,2,3,0],[2,3,2,1],[2,4,3,3]])
        b = column_matrix(R, [1,5,3])
        M_b = M.augment(b, subdivide=True)
      \end{sageblock}
      $$M_b = \sage{M_b}$$ con coeficientes en el campo $R$: \sage{R.__repr__()}.
      
      \hrulefill
      
      \item
      \begin{sageblock}
        x = polygen(GF(3), "x")
        R.<a> = GF(3).extension(x^2+1, "a")
        M = matrix(R, [[1,a,a+2],[2,1,0],[a,a,1]])
        b = column_matrix(R, [1,1,1])
        M_b = M.augment(b, subdivide=True)
      \end{sageblock}
      $$M_b = \sage{M_b}$$ con coeficientes en el campo $R$: \sage{R.__repr__()} donde $a$ es raíz de $\sage{a.minpoly()}$.
      
      \hrulefill
      
      \item
      \begin{sageblock}
        x = polygen(GF(3), "x")
        R.<a> = GF(3).extension(x^3+2*x+1, "a")
        M = matrix(R, [[1,a,a+2],[2,1,0],[a,a,1]])
        b = column_matrix(R, [1,1,1])
        M_b = M.augment(b, subdivide=True)
      \end{sageblock}
      $$M_b = \sage{M_b}$$ con coeficientes en el campo $R$: \sage{R.__repr__()} donde $a$ es raíz de $\sage{a.minpoly()}$.
      
    \end{enumerate}
    
    \item Determinar los polinomios característicos y los valores y vectores propios de cada una de las siguientes matrices. Extender, si es necesario, el campo de coeficientes.
    \begin{enumerate}
      \item
      \begin{sageblock}
        R = GF(5)
        M = matrix(R, [[1,2,2],[1,1,1],[3,2,1]])
      \end{sageblock}
      $$M = \sage{M}$$ con coeficientes en el campo $R$: \sage{R.__repr__()}.
      
      \hrulefill
      
      \item
      \begin{sageblock}
        R = GF(7)
        M = diagonal_matrix(R,[1,6,2,5,2])
        M.swap_rows(1,2)
        M.swap_rows(3,4)
        # Cuidado: swap_rows cuenta las filas desde el cero
      \end{sageblock}
      $$M = \sage{M}$$ con coeficientes en el campo $R$: \sage{R.__repr__()}.
      
      \hrulefill
      
      \item
      \begin{sageblock}
        x = polygen(GF(7), "x")
        R.<a> = GF(7).extension(x^3+2*x+1, "a")
        M = matrix(R, [[1,a,0,a+2],[0,a,a,1],
                       [1,0,5,3*a],[a,a^2,3,2]])
      \end{sageblock}
      
      $$M=\sage{M}$$ con coeficientes en el campo $R$: \sage{R.__repr__()} donde $a$ es raíz de $\sage{a.minpoly()}$.
      
    \end{enumerate}
    
    \item Mostrar: si $F$ es un campo finito, y $f:F\to F$ es un homomorfismo de anillos, entonces $f$ es un isomorfismo de anillos.
    
    \item Para cada campo finito $R$ indicado, se construye el conjunto $X$ de todos los homomorfismos de anillos $R\to R$. Listar y describir los elementos de $X$. Observando que la composición de dos elementos de $X$ es un elemento de $X$, construir la tabla de la operación composición entre estos isomorfismos. Finalmente, reconocer cuál de estos isomorfismos corresponde al morfismo de Frobenius.
    \begin{enumerate}
      \item
      \begin{sageblock}
        x = polygen(GF(7), "x")
        R.<a> = GF(7).extension(x^3+2*x+1, "a")
        X = End(R)
      \end{sageblock}
      
      \hrulefill
      
      \item
      \begin{sageblock}
        x = polygen(GF(13), "x")
        R.<a> = GF(13).extension(x^4+x^3+1, "a")
        X = End(R)
      \end{sageblock}
      
    \end{enumerate}
    
    \item En cada caso, determinar si la lista $B$ es una base de Gröbner en el anillo de polinomios $P$. Si no lo es, construir una base de Gröbner que genere el mismo ideal que el conjunto $B$.
    \begin{enumerate}
      \item
      \begin{sageblock}
        R = GF(13)
        P.<x,y,t> = PolynomialRing(R, order="lex")
        B = [x^2+y^2-1, y-t*(x+1)]
      \end{sageblock}
      
      Donde $P$: $\sage{P}$ y $B=\sage{B}$
      
      \hrulefill
      
      \item
      \begin{sageblock}
        x = polygen(GF(7), "x")
        R.<a> = GF(7).extension(x^2+1, "a")
        P.<x,y> = PolynomialRing(R, order="lex")
        B = [x^2+a*y^2+1, x*y+3*a]
      \end{sageblock}
      
      Donde $P$: $\sage{P}$ y $B=\sage{B}$
      
    \end{enumerate}
    
    \item Sean $E,F$ campos con $E\subseteq F$. Si $|E| = p^k$ y $|F|=p^l$, mostrar: $k\mid l$. \emph{(Indicación: considerar $F^+$ como un $E$-espacio vectorial... ¿qué relación hay entre su dimensión y su cardinalidad?)}
    
    \item Utilizando el problema anterior, describir todos los subcampos del campo con $64$ elementos \begin{sageblock}R.<a> = GF(2^6, "a")\end{sageblock}
    
    Para cada subcampo, determinar un generador.
    
    \item \emph{(Desafío de Software)} Construir un método general para obtener generadores de subcampos de un campo con $p^l$ elementos.



    
    \setcounter{problems}{\value{enumi}}
\end{enumerate}

\subsection*{Glosario de comandos útiles}

\textbf{Si $M$ es una matriz:}

\begin{itemize}
  \item \texttt{M.echelon\_form()} reduce la matriz a su forma reducida escalonada por filas.
  \item \texttt{M.inverse()} invierte la matriz.
  \item \texttt{M.right\_kernel()} entrega el kernel de $M$.
  \item \texttt{M.image()} entrega la imagen de $M$.
  \item \texttt{M.solve\_right(b)}, donde \texttt{b} es un vector columna, resuelve el sistema $Mx=b$ (sólo una solución).
  \item \texttt{M.eigenvalues()} entrega una lista con los valores propios de $M$.
  \item \texttt{M.charpoly()}, \texttt{M.minpoly()} entrega, respectivamente, los polinomios característico y minimal de $M$.
\end{itemize}

\textbf{Si $E$, $F$ son campos:}

\begin{itemize}
  \item \texttt{E.frobenius\_endomorphism()} entrega el morfismo de Frobenius $E\to E$.
  \item \texttt{Hom(E,F)} construye el conjunto de todos los homomorfismos de campos de $E$ en $F$.
  \item \texttt{V,f,g = E.vector\_space()} construye tres objetos: el espacio vectorial $V$ de las coordenadas de $E^+$ en la base canónica $\{1,a,a^2,\ldots ,a^{d-1}\}$, el isomorfismo de espacios vectoriales $f:V\to E^+$, y su isomorfismo inverso $g:E^+\to V$.
\end{itemize}








\end{document}