\documentclass[10pt,a4paper]{report}
\usepackage[utf8]{inputenc}
\usepackage[spanish]{babel}
\usepackage[T1]{fontenc}
\usepackage{amsmath}
\usepackage{amsfonts}
\usepackage{amssymb}
\usepackage{amsthm}
\usepackage{stmaryrd}
\usepackage{stix}
\usepackage[mathscr]{euscript}
% Bibliografia
\usepackage[backend=biber]{biblatex}
\bibliography{Referencias.bib}
% Cajas
\usepackage{tcolorbox}
\usepackage{tikz}
% Comandos
\newtheorem{teo}{Teorema}
\newtheorem{prop}{Proposición}
\newtheorem{pro}{Problema}
\newcommand{\autor}{\textbf{Brayan Sandoval}}
\newcommand{\asignatura}{\textbf{Elementos Finitos}}
\newcommand{\tarea}{\textbf{Tarea 1}}
\newcommand{\fecha}{\textbf{\today}}
\newcommand{\bs}{\boldsymbol}
\newcommand{\dx}{\textup{d}x}
\newcommand{\dy}{\textup{d}y}
\newcommand{\dt}{\textup{d}t}
\newcommand{\ds}{\textup{d}s}
\newcommand{\dS}{\textup{d}S}
\newcommand{\DG}{\textup{DG}}
\newcommand{\Th}{\mathscr{T}_{h}}
\newcommand{\Hdiv}{H\left(\textup{div};\Omega\right)}
\newcommand{\Hdivt}{H(\textup{div};\mathscr{T}_{h})}
\providecommand{\Dt}[1]{\frac{\textup{d} #1}{\textup{d}t}}
\providecommand{\Dx}[1]{\frac{\textup{d} #1}{\textup{d}x}}
\providecommand{\abs}[1]{\left\lvert#1\right\rvert}
\providecommand{\norm}[1]{\left\lVert#1\right\rVert}
\providecommand{\Norm}[1]{\lVert#1\rVert}
\providecommand{\salto}[1]{\left\llbracket#1\right\rrbracket}
\providecommand{\prom}[1]{\left \{\!\left \{#1\right \}\!\right \}}
\providecommand{\PI}[2]{\left\langle #1,#2  \right\rangle}
\providecommand{\Pii}[2]{\left( #1,#2  \right)}
\renewcommand{\theequation}{\roman{equation}}

%texto
\usepackage{lipsum} % Genera texto aleatorio
\renewcommand*{\familydefault}{\sfdefault} % Letra mas bonita
% Figuras
\usepackage{graphicx}
% Geometría
\usepackage[left= 2 cm, right = 2 cm, top = 2 cm, bottom = 2 cm]{geometry}
\usepackage{lastpage}
% Color
\usepackage{xcolor}
\definecolor{azul}{RGB}{10,10,115}
\definecolor{amarillo}{RGB}{255,204,0}
\definecolor{rojo}{RGB}{247,0,30}
% Encabezado
\usepackage{fancyhdr}
\pagestyle{fancy}
\renewcommand{\headrulewidth}{4pt} %Aumentar grosor linea encabezado
\let\oldheadrule\headrule
\renewcommand{\headrule}{\color{azul}\oldheadrule}
\renewcommand{\footrulewidth}{4pt} %Aumentar grosor linea pie de pagina
\let\oldfootrule\footrule
\renewcommand{\footrule}{\color{azul}\oldfootrule}
\rhead{\color{azul}\autor}
\chead{\color{azul}\tarea}
\lhead{\color{azul}\asignatura}
\rfoot{\color{azul} \textbf{Pág. \thepage\ - \pageref{LastPage}}}
\cfoot{}
\lfoot{\color{azul}\fecha}
% Titulo
\title{\color{azul}\textbf{Análisis Real I }\\
	\textbf{Técnicas de Demostración}}
\author{\color{azul}\autor}
\date{\color{azul}\fecha}
\begin{document}
	\begin{pro}
		Considere el espacio $V_{h} := \left\{ v\in \mathcal{C}(a,b):\ v\in\mathbb{P}_{1}([x_{i-1},x_{i}]), \text{para}\ i = 1,\dots,d+1,\ v(a) = v(b) = 0\right \}$ y la familia de funciones techo $B := \{\varphi_{1},\varphi_{2},\dots, \varphi_{d}\}$ definidas en clase. Demuestre que $B$ es una base de $V_{h}$.
	\end{pro}
    \begin{proof}
        Sea ver que $B$ es una base, es necesario probar 
        \begin{itemize}
        	\item $B$ es linealmente independiente. Sea $\{\alpha_{i}\}^{d}_{i=1}\subset\mathbb{R}$ tal que 
        	\begin{align*}
        		\sum_{i=1}^{d}\alpha_{i}\varphi_{i} = \theta,
        	\end{align*}
            como $\varphi_{i}(x_{j})=\delta_{ij}$ se tiene que 
            \begin{align*}
            	0 =\sum_{i=1}^{d}\alpha_{i}\varphi_{i}(x_{j}) =  \sum_{i=1}^{d}\alpha_{i}\delta_{ij} = a_{j},\ \forall j\in\{1,\dots,d\}.
            \end{align*}
             Por tanto, $B$ es linealmente independiente.
             \item $\langle B \rangle = V_{h}$. Por doble inclusión
             \begin{itemize}
             	\item[$(\subseteq)$] Dado $i\in\{1,\dots,d\}$ es inmediato notar que $\varphi_{i}\in V_{h}$. Luego, como $V_{h}$ es un sub espacio vectorial, se tiene que $\langle B \rangle \subseteq V_{h}$.
             	\item[$(\supseteq)$] Dado $v\in V_{h}$, se define $v_{I}$ como sigue
             	\begin{align*}
             		v_{I} = \sum_{i=1}^{d}v(x_{i})\varphi_{i}
             	\end{align*}
                A continuación probaremos que $v - v_{I} = \theta$ lo cual probaría la inclusión deseada. Dado $j\in\{0,\dots,d\}$ se tiene
                \begin{align*}
                	\left(v - v_{I}\right)(x_{j}) = v(x_{j}) - v_{I}(x_{j}) = v(x_{j}) - \sum_{i=1}^{d}v(x_{i})\varphi_{i}(x_{j}) = v(x_{j}) -  \sum_{i=1}^{d}v(x_{i})\delta_{ij} = v(x_{j}) - v(x_{j}) = 0 
                \end{align*} 
                y como $(v - v_{I})\in \mathbb{P}\left[x_{j-1},x_{j}\right]$ se deduce que necesariamente $(v - v_{I})(x) = 0$ para todo $x\in \left[x_{j-1},x_{j}\right]$. Como lo anterior es valido para todo $j\in\{1,\dots,d+1\}$ se deduce que $(v - v_{I}) = \theta$, de esta forma 
                \begin{align*}
                	v = \sum_{i=1}^{d}v(x_{i})\varphi_{i}
                \end{align*}
                es decir, $v\in \langle B \rangle$. Lo cual prueba que $\langle B \rangle \supseteq V_{h}$.

%             	veamos que existe $\{\beta_{i}\}_{i=1}^{d}\subset\mathbb{R}$ tal que 
%             	\begin{align*}
%             		v = \sum_{i=1}^{d} \beta_{i}\varphi_{i}.
%             	\end{align*} 
%                Notando que $\varphi_{i}(x_{j}) = \delta_{ij}$, para todo $i,j\in\{1,\dots,d\}$, se deduce
%                \begin{align*}
%                	v(x_{j}) = \sum_{i=1}^{d}\beta_{i}\varphi_{i}(x_{j}) = \sum_{i=1}^{d}\beta_{i}\delta_{ij} = \beta_{j}
%                \end{align*}
%                 por lo que 
%                 \begin{align*}
%                 	v = \sum_{i=1}^{d}v(x_{i})\varphi_{i}
%                 \end{align*}
             \end{itemize}
              Como ambas inclusiones se cumplen se deduce que $\langle B\rangle = V_{h}$.  
        \end{itemize}
        Finalmente, se tiene que $B$ es base de $V_{h}$.
    \end{proof}
    \begin{pro}
    	Considere la forma bilineal $\displaystyle a(u_{h},v_{h}) = \int_{a}^{b}u'_{h}v'_{h}\text{d}x$, su matriz $\boldsymbol{A}$ asociada y el espacio $V_{h}$ definido e el Problema 1.
    	\begin{itemize}
    		\item[$a)$] Sea $u_{h}\in V_{h}$. Sabemos que existen escalares $\beta_{1},\beta_{2},\dots,\beta_{d}$ tales que $\displaystyle v_{h}(x) = \sum_{j=1}^{d}\beta_{j}\varphi_{j}(x)$. Probar\\ que $\boldsymbol{\beta}^{t}\boldsymbol{A}\boldsymbol{\beta} = \Norm{v'_{h}}^{2}_{L^{2}(a,b)}$, donde $\boldsymbol{\beta}$ es el vector cuyos elementos son los coeficientes $\beta_{j}$.
    		\item[$b)$] Demostrar que $\boldsymbol{A}$ es simétrica y definida positiva.
    	\end{itemize}
    \end{pro}
    \begin{proof}
    	\begin{itemize}
    		\item[$a)$] Dados $\beta_{1},\beta_{2},\dots,\beta_{d}$ tales que 
    		\begin{align}
    			v_{h}(x) = \sum_{j=1}^{d}\beta_{j}\varphi_{j}(x),\ \forall x\in(a,b).\label{vh}
    		\end{align}
    	    De un calculo directo se tiene que
    	    \begin{align*}
    	    	\Norm{v'_{h}}^{2}_{L^{2}(a,b)} = \int_{a}^{b}v'_{h}v'_{h}\text{d}x = a(v_{h},v_{h})
    	    \end{align*}
            usando \eqref{vh} y que $a$ es una forma bilineal se tiene
            \begin{align*}
            	a(v_{h},v_{h}) &= a\left( \sum_{i=1}^{d}\beta_{i}\varphi_{i},\sum_{j=1}^{d}\beta_{j}\varphi_{j}\right)
            	               =\sum_{j=1}^{d}\beta_{j}a\left(\sum_{i=1}^{d}\beta_{i}\varphi_{i},\varphi_{j}\right)
            	               =\sum_{j=1}^{d}\sum_{i=1}^{d}\beta_{j}\beta_{i}a(\varphi_{i},\varphi_{j}) = \sum_{j=1}^{d}\beta_{j}\sum_{i=1}^{d}\frac{a_{ij}}{h}\beta_{i}
            	               = \boldsymbol{\beta}^{t}\boldsymbol{A}\boldsymbol{\beta}
            \end{align*}
            \item[$b)$] Recordando que 
            \begin{align*}
            	\boldsymbol{A} = \frac{1}{h}\begin{pmatrix}
            		2 & -1 &   & \\ 
            		-1 & \ddots  & \ddots  & \\ 
            		& \ddots  & \ddots  & -1\\ 
            		&  & -1 & 2
            	\end{pmatrix}
            \end{align*}
            Es directo notar que $\boldsymbol{A} = \boldsymbol{A}^{t}$. Por otro lado, dado $\boldsymbol{\beta} = \left(\beta_{1},\dots,\beta_{d}\right)^{t}$ un vector  no nulo de $\mathbb{R}^{d}$ fijo pero arbitrario, se tiene que existe una función $\omega_{h}\in V_{h}$ tal que 
            \begin{align*}
            	\omega_{h} = \sum_{i=1}^{d}\beta_{i}\varphi_{i}.
            \end{align*}
            Luego, por $a)$ se tiene que 
            \begin{align*}
            	\boldsymbol{\beta}^{t}\boldsymbol{A}\boldsymbol{\beta} = \Norm{\omega'_{h}}^{2}_{L^{2}(a,b)} > 0. 
            \end{align*}
            Como lo anterior es válido para cualquier vector no nulo de $\mathbb{R}^{d}$ se concluye que $\boldsymbol{A}$ es definida positiva.
            
     	\end{itemize}
    \end{proof}
    \begin{pro}
    	Considere la ecuación
    	\begin{align*}
    		\left\{\begin{aligned}
    			-(\kappa(x)u'(x))' + \omega u(x) &= f(x),\ x\in]a,b[\\ 
    			u(a) &= 0,\\ 
    			u(b) &= 0,
    		\end{aligned}
    		\right.
    	\end{align*}
        con $f\in L^{2}(a,b)$, $\omega$ y $\kappa$ dados. Asuma que la función $\kappa$ es continua y positiva en $[a,b]$, y $\omega$ es un número positivo.
        \begin{itemize}
        	\item[$a)$] Deduzca una formulación variacional apropiada.
        	\item[$b)$] Demuestre que dicha formulación variacional posee solución única.
        	\item[$c)$] Escriba la formulación variacional discreta asociada considerando una partición uniforme $\{x_{i}\}_{i=0}^{d+1}$ de tamaño $h$, del intervalo $[0,1]$ y el espacio 
        	\begin{align*}
        		V_{h} := \{v\in\mathcal{C}(a,b):\ v\in \mathbb{P}\left(\left[x_{i-1},x_{i}\right]\right), \ \text{para}\ i = 1,\dots,d+1,\ v(a) = v(b) = 0 \}.
        	\end{align*}
            \item[$d)$] Escriba el sistema lineal $\boldsymbol{A}\boldsymbol{\alpha} = \boldsymbol{b}$ asociado.
        \end{itemize}
    \end{pro}
    \begin{proof}
    	\begin{itemize}
    		\item[$a)$] Testeando la EDO con una función $v\in H_{0}^{1}(a,b)$ se obtiene 
    		\begin{align}
    			-\int_{a}^{b}(\kappa(x)u'(x))'v(x)\ \text{d}x + \omega\int_{a}^{b}u(x)v(x)\ \text{d}x = \int_{a}^{b}f(x)v(x)\ \text{d}x.\label{testeo}
    		\end{align}
    	    Haciendo integración por partes al primer termino del lado izquierdo de \eqref{testeo} y usando las condiciones de contorno se obtiene 
    	    \begin{align*}
    	    	\int_{a}^{b} u'(x)v'(x)\kappa(x)\ \text{d}x +  \omega\int_{a}^{b}u(x)v(x)\ \text{d}x = \int_{a}^{b}f(x)v(x)\ \text{d}x.
    	    \end{align*}
            De lo anterior se tiene la forma bilineal $a:H_{0}^{1}(a,b)\times H_{0}^{1}(a,b)\longrightarrow \mathbb{R}$ y el funcional lineal $F:H_{0}^{1}(a,b)\longrightarrow\mathbb{R}$ definidos por
            \begin{align*}
            	a(u,v) := \int_{a}^{b} u'(x)v'(x)\kappa(x)\ \text{d}x +  \omega\int_{a}^{b}u(x)v(x)\ \text{d}x
            \end{align*}
            y 
            \begin{align*}
            	F(v) := \int_{a}^{b}f(x)v(x)\ \text{d}x.
            \end{align*}
            La linealidad de $a$ y $F$ es consecuencia directa de que la integral y la derivada son operadores lineales. Finalmente, el problema queda
            \begin{align*}
            	\left\{\begin{aligned}
            		&\text{Hallar}\ u\in H_{0}^{1}(a,b)\  \text{tal que}\\
            		 &\ \ \ \ \  a(u,v) = F(v)\\ 
            		&\text{para todo}\ v\in H_{0}^{1}(a,b).
            	\end{aligned}
            	\right.
            \end{align*}
            \item[$b)$] Para probar existencia unicidad se debe probar 
            \begin{itemize}
            	\item[$i)$] $a$ es una forma bilineal acotada.\\
            	Dados $u,v\in V := H_{0}^{1}(\Omega)$, luego 
            	\begin{align*}
            		|a(u,v)| &= \left|\int_{a}^{b} u'(x)v'(x)\kappa(x)\ \text{d}x +  \omega\int_{a}^{b}u(x)v(x)\ \text{d}x\right|\\ &\leq  \left|\int_{a}^{b} u'(x)v'(x)\kappa(x)\ \text{d}x\right| +  \omega\left|\int_{a}^{b}u(x)v(x)\ \text{d}x\right|\\ &\leq \norm{u'}_{L^{2}(a,b)}\norm{v'\kappa}_{L^{2}(a,b)} + \omega\norm{u}_{L^{2}(a,b)}\norm{v}_{L^{2}(a,b)}\\ &\leq M\norm{u'}_{L^{2}(a,b)}\norm{v'}_{L^{2}(a,b)} + \omega\norm{u}_{L^{2}(a,b)}\norm{v}_{L^{2}(a,b)}\\
            		&\leq M \norm{u}_{H^{1}(a,b)}\norm{v}_{H^{1}(a,b)} + \omega\norm{u}_{H^{1}(a,b)}\norm{v}_{H^{1}(a,b)}\\
            		&\leq \max\{M,\omega\}\norm{u}_{H^{1}(a,b)}\norm{v}_{H^{1}(a,b)}.
            	\end{align*}
                Donde $M$ resulta de usar el hecho que $\kappa$ es continua en un compacto. Así, $a$ es una forma bilineal acotada.
            	\item[$ii)$] $a$ es coerciva.\\
            	Dado $v\in V$, luego
            	\begin{align*}
            		a(v,v) &=  \int_{a}^{b} v'(x)v'(x)\kappa(x)\ \text{d}x +  \omega\int_{a}^{b}v(x)v(x)\ \text{d}x\\ &\geq m\int_{a}^{b} v'(x)v'(x)\ \text{d}x + \omega\int_{a}^{b}v(x)v(x)\ \text{d}x\\ &\geq \min\{m,\omega\}\norm{v}^{2}_{H^{1}(a,b)}.
            	\end{align*}
                Donde $m$ resulta de usar el hecho que $\kappa$ es continua en un compacto.
                Así $a$ es coreciva.
            	\item[$iii)$] $F$ es un funcional lineal y acotado.\\
            	Dado $v\in V$, luego
            	\begin{align*}
            		\left|\int_{a}^{b} f(x)v(x)\ \text{d}x\right| \leq \norm{f}_{L^{2}(a,b)}\norm{v}_{L^{2}(a,b)}\leq \norm{f}_{L^{2}(a,b)}\norm{v}_{H^{1}(a,b)}
            	\end{align*}
                Así, $F$ es un funcional lineal y acotado.
            	\item[$iv)$] $\left(H_{0}^{1}(\Omega),\PI{\cdot}{\cdot}\right)$ es un espacio de Hilbert.\\ 
            	Directo del hecho de que $H_{0}^{1}(\Omega)$ es un subespacio cerrado de $H^{1}(a,b)$.
            \end{itemize}
            Finalmente como se cumplen las hipótesis del Lema de Lax-Milgram se deduce que  \begin{align*}
            	\left\{\begin{aligned}
            		&\text{Existe un único}\ u\in H_{0}^{1}(a,b)\  \text{tal que}\\
            		&\ \ \ \ \  a(u,v) = F(v)\\ 
            		&\text{para todo}\ v\in H_{0}^{1}(a,b).
            	\end{aligned}
            	\right.
            \end{align*}
            y ademas, 
            \begin{align*}
            	\norm{u}_{H^{1}(a,b)}\leq\frac{1}{\min\{m,\omega\}}\norm{F}\leq \frac{1}{\min\{m,\omega\}}\norm{f}_{H^{1}(a,b)}.
            \end{align*}
            \item[$c)$] Considerando la partición uniforme del enunciado y el espacio
            	\begin{align*}
            	    V_{h} := \{v\in\mathcal{C}(a,b):\ v\in \mathbb{P}\left(\left[x_{i-1},x_{i}\right]\right), \ \text{para}\ i = 1,\dots,d+1,\ v(a) = v(b) = 0 \}.
                \end{align*}
                Se tiene el problema discreto
                \begin{align*}
                	\left\{\begin{aligned}
                		&\text{Hallar}\ u_{h}\in V_{h}\  \text{tal que}\\
                		&\ \ \ \ \  a(u_{h},v_{h}) = F(v_{h})\\ 
                		&\text{para todo}\ v_{h}\in V_{h}.
                	\end{aligned}
                	\right.
                \end{align*}
                el cual es equivalente a 
                \begin{align}\label{problemadiscreto}
                	\left\{\begin{aligned}
                		&\text{Hallar}\ u_{h}\in V_{h}\  \text{tal que}\\
                		&\ \ \ \ \  a(u_{h},\varphi_{j}) = F(\varphi_{j})\\ 
                		&\text{para todo}\ j\in\{1,\dots,d\}.
                	\end{aligned}
                	\right.
                \end{align}
            \item[$d)$] Usando que $\{\varphi_{1},\varphi_{2},\dots,\varphi_{d}\}$ es una base de $V_{h}$ se tiene que existe $\boldsymbol{\alpha} = \left(\alpha_{1},\alpha_{2},\dots,\alpha_{d}\right)^{t}\in\mathbb{R}^{d}$ tal que 
            \begin{align*}
            	u_{h} = \sum_{i=1}^{d}\alpha_{i}\varphi_{i}
            \end{align*}
            insertando esto en \eqref{problemadiscreto} se obtiene el sistema de ecuaciones
            \begin{align*}
            	\sum_{i=1}^{d}\alpha_{i}a(\varphi_{i},\varphi_{j}) = F(\varphi_{j})
            \end{align*}
            donde la incógnita es $\boldsymbol{\alpha}$ de esta forma la matriz $\boldsymbol{A}$ cumple que $a_{ij} = a(\varphi_{i},\varphi_{j})$ cuando $i,j\in\{1,\dots,d\}$. Notando que $\text{supp}(\varphi_{i})\cap\text{supp}(\varphi_{i+2})=\emptyset$ se deduce que $a_{i,i+l} = 0$ para todo $i\in\{1,\dots,d-l\}$ con $l>1$, de la misma forma se concluye que $a_{i,i-l} = 0$. Por otro lado, para el calculo de los demás coeficientes se procede como sigue
            \begin{itemize}
            	\item Cálculo de $a_{i,i}$ con $i\in\{1,\dots,d\}$.
            	\begin{align*}
            		a_{i,i}
            		 &= a(\varphi_{i},\varphi_{i})\\ &= \int_{a}^{b}\varphi'_{i}(x)\varphi'_{i}(x)\kappa(x)\ \text{d}x + \omega\int_{a}^{b}\varphi_{i}(x)\varphi_{i}(x)\ \text{d}x\\ &= \int_{x_{i-1}}^{x_{i}}\varphi'_{i}(x)\varphi'_{i}(x)\kappa(x)\ \text{d}x + \int_{x_{i}}^{x_{i+1}}\varphi'_{i}(x)\varphi'_{i}(x)\kappa(x)\ \text{d}x + \omega\int_{x_{i-1}}^{x_{i}}\varphi_{i}(x)\varphi_{i}(x)\ \text{d}x + \omega\int_{x_{i}}^{x_{i+1}}\varphi_{i}(x)\varphi_{i}(x)\ \text{d}x\\
            		 &= \int_{x_{i-1}}^{x_{i}}\frac{1}{h^{2}}\kappa(x)\ \text{d}x + \int_{x_{i}}^{x_{i+1}}\frac{1}{h^{2}}\kappa(x)\ \text{d}x + \omega\int_{x_{i-1}}^{x_{i}}\varphi_{i}(x)\varphi_{i}(x)\ \text{d}x + \omega\int_{x_{i}}^{x_{i+1}}\varphi_{i}(x)\varphi_{i}(x)\ \text{d}x\\
            		 &= \int_{x_{i-1}}^{x_{i}}\frac{1}{h^{2}}\kappa(x)\ \text{d}x + \int_{x_{i}}^{x_{i+1}}\frac{1}{h^{2}}\kappa(x)\ \text{d}x + \omega\int_{x_{i-1}}^{x_{i}}\frac{(x-x_{i})^{2}}{h^{2}}\ \text{d}x + \omega\int_{x_{i}}^{x_{i+1}}\frac{(x_{i+1}-x)^{2}}{h^{2}}\ \text{d}x\\
            		 &= \int_{x_{i-1}}^{x_{i}}\frac{1}{h^{2}}\kappa(x)\ \text{d}x + \int_{x_{i}}^{x_{i+1}}\frac{1}{h^{2}}\kappa(x)\ \text{d}x + \frac{h}{3}\omega - \frac{h}{3}\omega \\
            		 &\approx \frac{1}{h}\left[\kappa\left(\frac{x_{i-1}+x_{i}}{2}\right) + \kappa\left(\frac{x_{i}+x_{i+1}}{2}\right)\right] 
            	\end{align*}
                 \item Cálculo de $a_{i,i+1}$ con $i\in\{1,\dots,d-1\}$.
                 \begin{align*}
                 	a_{i,i+1} &= a(\varphi_{i},\varphi_{i+1})\\
                 	&= \int_{a}^{b}\varphi'_{i}(x)\varphi'_{i+1}(x)\kappa(x)\ \text{d}x + \omega\int_{a}^{b}\varphi_{i}(x)\varphi_{i+1}(x)\ \text{d}x\\
                 	&= \int_{x_{i}}^{x_{i+1}}\frac{-1}{h^{2}}\kappa(x)\ \text{d}x + \omega\int_{x_{i}}^{x_{i+1}}\frac{(x_{i+1}-x)}{h}\frac{(x-x_{i+1})}{h}\ \text{d}x\\
                 	&= \frac{-1}{h^{2}}\int_{x_{i}}^{x_{i+1}}\kappa(x)\ \text{d}x  -\frac{\omega}{h^{2}}\int_{x_{i}}^{x_{i+1}}(x-x_{i+1})^{2}\ \text{d}x\\
                 	&=  \frac{-1}{h^{2}}\int_{x_{i}}^{x_{i+1}}\kappa(x)\ \text{d}x  - \frac{h\omega}{3}\\
                 	&\approx \frac{-1}{h}\kappa\left(\frac{x_{i}+x_{i+1}}{2}\right) - \frac{h\omega}{3}
                 \end{align*} 
                 \item Cálculo de $a_{i,i-1}$ con $i\in\{2,\dots,d+1\}$.
                 \begin{align*}
                 	a_{i,i-1} &= a(\varphi_{i},\varphi_{i-1})\\
                 	&= \int_{a}^{b}\varphi'_{i}(x)\varphi'_{i-1}(x)\kappa(x)\ \text{d}x + \omega\int_{a}^{b}\varphi_{i}(x)\varphi_{i-1}(x)\ \text{d}x\\
                 	&= \int_{x_{i-1}}^{x_{i}}\frac{-1}{h^{2}}\kappa(x)\ \text{d}x + \omega\int_{x_{i-1}}^{x_{i}}\frac{(x-x_{i})}{h}\frac{(x_{i}-x)}{h}\ \text{d}x\\
                 	&= \frac{-1}{h^{2}}\int_{x_{i-1}}^{x_{i}}\kappa(x)\ \text{d}x  -\frac{\omega}{h^{2}}\int_{x_{i-1}}^{x_{i}}(x-x_{i})^{2}\ \text{d}x\\
                 	&=  \frac{-1}{h^{2}}\int_{x_{i-1}}^{x_{i}}\kappa(x)\ \text{d}x  - \frac{h\omega}{3}\\
                 	&\approx \frac{-1}{h}\kappa\left(\frac{x_{i-1}+x_{i}}{2}\right) - \frac{h\omega}{3}
                 \end{align*} 
            \end{itemize}
             Las aproximaciones al final se deben al hecho de que se uso la regla del punto medio elemental para aproximar las integrales de la función $\kappa$.
             Para determinar el vector $\boldsymbol{b}\in\mathbb{R}^{d}$ se procede de forma similar, dado $j\in\{1,\dots,d\}$ luego
             \begin{align*}
             	b_{j} &= F(\varphi_{j})\\ &= \int_{a}^{b} f(x)\varphi_{j}(x)\text{d}x\\
             	&= \int_{x_{j-1}}^{x_{j+1}}f(x)\varphi_{j}(x)\text{d}x\\
             	&= \int_{x_{j-1}}^{x_{j}}f(x)\frac{x-x_{j}}{h}\text{d}x + \int_{x_{j}}^{x_{j+1}}f(x)\frac{x_{j+1}-x}{h}\text{d}x\\
             	&\approx \frac{h}{2}\left[f\left(\frac{x_{j}+x_{j-1}}{2}\right) + f\left(\frac{x_{j}+x_{j+1}}{2}\right)\right]
             \end{align*}
    	\end{itemize}
    \end{proof}
\end{document}