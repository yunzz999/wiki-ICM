\documentclass[11pt]{article}

\textwidth 17cm
\textheight 23cm
\oddsidemargin 0.25cm
\addtolength{\voffset}{-2.4cm}
\addtolength{\hoffset}{-0.5cm}

\setlength{\parindent}{12pt}
\setlength{\parskip}{3pt}
\usepackage{amssymb,amsmath,epsfig}
\usepackage[colorlinks=false,breaklinks=true,linkcolor=blue]{hyperref}
\usepackage{graphics,psfrag,graphicx,color}
\usepackage{diagbox}
\usepackage{float,hhline}

%\usepackage{refcheck}

\pdfminorversion=7

% OUR DEFINITIONS %%%%%%%%%%%%%%%%%%%%%%%%%%%%%%%%%%%

\newcommand{\ds}{\displaystyle}
\newcommand{\bomega}{{\boldsymbol\omega}}
\newcommand{\bsi}{{\boldsymbol\sigma}}
\newcommand\bta{{\boldsymbol\tau}}
\newcommand{\bze}{{\boldsymbol\zeta}}
\newcommand\bet{{\boldsymbol\eta}}
\newcommand{\bv}{{\mathbf{v}}}
\newcommand{\bw}{{\mathbf{w}}}
\newcommand\bz{{\mathbf z}}
\newcommand{\f}{\mathbf{f}}
\newcommand{\g}{\mathbf{g}}
\newcommand{\bu}{\mathbf{u}}
\newcommand{\bn}{{\mathbf{n}}}
\newcommand\bbH{{\mathbb H}}
\newcommand\bbL{{\mathbb L}}
\newcommand\bL{{\mathbf L}}
\renewcommand\div{{\mathrm{div}}}
\newcommand\R{{\mathrm R}}
\newcommand\Q{{\mathrm Q}}
\newcommand\X{{\mathrm X}}
\newcommand\bG{{\mathbf G}}
\newcommand\bF{{\mathbf F}}
\newcommand\A{{\mathrm A}}
\newcommand\B{{\mathrm B}}
\newcommand\F{{\mathrm F}}
\newcommand\Y{{\mathrm Y}}
\newcommand\G{{\mathrm G}}

\def\bI{\mathbb{I}}
\def\bM{\mathbf{M}}
\newcommand\bH{\mathbf{H}}
\newcommand\bbM{\mathbb{M}}
\newcommand\bbI{\mathbb{I}}
\newcommand\bt{{\mathbf t}}
\newcommand\ba{{\mathbf a}}
\newcommand\bb{{\mathbf b}}
\newcommand\bc{{\mathbf c}}

\newcommand\qin{{\quad\hbox{in}\quad}}
\newcommand\qon{{\quad\hbox{on}\quad}}

\def\R{\mathrm{R}}
\def\H{\mathrm{H}}
\def\L{\mathrm{L}}
\def\M{\mathrm{M}}
\def\W{\mathrm{W}}
\def\rd{\mathrm{d}}
\def\rD{\mathrm{D}}
\def\rt{\mathrm{t}}
\def\tF{{\mathtt F}}
\def\tD{{\mathtt D}}
\def\bdiv{\mathbf{div}}

\def\tr{\mathrm{tr}}
\def\pil{\left<}
\def\pir{\right>}
\def\qan{{\quad\hbox{and}\quad}}
\def\wt{\widetilde}

\newtheorem{thm}{Theorem}[section]
\newtheorem{rem}{Remark}[section]
\newtheorem{lem}[thm]{Lemma}
\newtheorem{exa}{Example}[section]
\newtheorem{cor}[thm]{Corollary}
\newtheorem{defi}{Definition}[section]
\newtheorem{prop}{Proposition}[section]
\newenvironment{proof}{\noindent{\it Proof.}}{\hfill$\square$}

\numberwithin{equation}{section}
\numberwithin{figure}{section}
\numberwithin{table}{section}

% END OF OUR DEFINITIONS 

\allowdisplaybreaks

%***********************************************************************************
\title{A Banach spaces-based mixed-primal finite element method for the coupled
Brinkman-Forchheimer and nonlinear transport equations\thanks{This research was supported by 
ANID-Chile through the projects {\sc Centro de Modelamiento Matem\'atico} (FB210005),
{\sc Anillo of Computational Mathematics for Desalination Processes} (ACT210087),
and Fondecyt 11220393; by Grupo de Investigaci\'on en An\'alisis Num\'erico y C\'alculo Cient\'ifico (GIANuC$^2$), 
Universidad Cat\'olica de la Sant\'isima Concepci\'on; and by Centro de Investigaci\'on en Ingenier\'ia Matem\'atica 
(CI$^2$MA), Universidad de Concepci\'on.}}

\author{
{\sc Fernando A. Artaza}\thanks{CI$^2$MA and Departamento de Ingenier\'ia Matem\'atica, 
Universidad de Concepci\'on, Casilla 160-C, Concepci\'on, Chile, email: {\tt fartaza2019@udec.cl}.}
\quad
{\sc Sergio Caucao}\thanks{GIANuC$^2$ and Departamento de Matem\'atica y F\'isica Aplicadas, Universidad Cat\'olica de la Sant\'isima Concepci\'on, Casilla 297, Concepci\'on, Chile,
email: {\tt scaucao@ucsc.cl}.}
\\
{\sc Gabriel N. Gatica}\thanks{CI$^2$MA and Departamento de Ingenier\'ia Matem\'atica, 
Universidad de Concepci\'on, Casilla 160-C, Concepci\'on, Chile, email: {\tt ggatica@ci2ma.udec.cl}.}
\quad
{\sc Brayan R. Sandoval}\thanks{CI$^2$MA and Departamento de Ingenier\'ia Matem\'atica, 
Universidad de Concepci\'on, Casilla 160-C, Concepci\'on, Chile, email: {\tt bsandoval2018@udec.cl}.}}

\date{ }

\begin{document}

\maketitle

\begin{abstract}
\noindent 
We propose and analyze a Banach spaces-based mixed-primal finite element method for the 
coupling of the Brinkman--Forchheimer equations with a nonlinear transport phenomenon 
 
 ...............
 
 
\end{abstract}

\noindent
{\bf Key words}: Brinkman--Forchheimer equations, nonlinear transport,
pseudostress-velocity \\
formulation, fixed point theory, perturbed saddle-point, mixed finite elements

\smallskip\noindent
{\bf Mathematics subject classifications (2000)}: 65N30, 65N12, 65N15, 35Q79, 80A20, 76R05, 76D07

\maketitle

%************************************************************************
%************************************************************************


\section{Introduction}\label{section1}


\subsection*{Preliminary notations}

Let $\Omega\subset \R^n, n\in\{2,3\}$, be a bounded domain with polyhedral boundary $\Gamma$, 
and let $\bn$ be the outward unit normal vector on $\Gamma$. In what follows, standard notation is adopted 
for Lebesgue spaces $\L^p(\Omega)$ and Sobolev spaces $\W^{s,p}(\Omega)$, with $s\in \R$ and $p > 1$, 
whose corresponding norms, either for the scalar, vectorial, or tensorial case, are denoted by 
$\|\cdot\|_{0,p;\Omega}$ and $\|\cdot\|_{s,p;\Omega}$, respectively. In particular, given a non-negative 
integer $m$, $\mathrm W^{m,2}(\Omega)$  is also denoted by $\mathrm H^m(\Omega)$, and the 
notations of its norm and seminorm are simplified to $||\cdot||_{m,\Omega}$ and $|\cdot|_{m,\Omega}$, 
respectively. In addition, $\H^{1/2}(\Gamma)$ is the space of traces of functions of $\H^1(\Omega)$, and 
$\H^{-1/2}(\Gamma)$ denotes its dual. On the other hand, given any generic scalar functional space 
$\mathrm S$, we let $\mathbf{S}$ and $\mathbb{S}$ be the corresponding vectorial and tensorial 
counterparts, whereas $\| \cdot\|$, with no subscripts, will be employed for the norm of any element or 
operator whenever there is no confusion about the space to which they belong.
Also, $|\cdot|$ denotes the Euclidean norm in both $\mathrm R^n$ and $\mathrm R^{n\times n}$, 
and as usual, $\mathbb I$ stands for the identity tensor in $\mathrm R^{n\times n}$.
In turn, for any vector fields $\bv=(v_i)_{i=1,n}$ and $\bw=(w_i)_{i=1,n}$, we set the gradient,  
divergence, and tensor product operators, as
\begin{equation*}
\nabla\bv:=\left(\frac{\partial v_i}{\partial x_j}\right)_{i,j=1,n},\quad 
\div (\bv):=\sum_{j=1}^n \frac{\partial v_j}{\partial x_j},\qan 
\bv\otimes\bw:=(v_i w_j)_{i,j=1,n}\,,
\end{equation*}
whereas for any tensor fields $\bta =(\tau_{ij})_{i,j=1,n}$ 
and $\bze=(\zeta_{ij})_{i,j=1,n}$, we let $\bdiv(\bta )$ be the divergence operator $\div$ acting 
along the rows of $\bta $, and define the transpose, the trace, the 
deviatoric tensor, and the tensor inner product, respectively, as
\begin{equation*}
\bta ^\rt := (\tau_{ji})_{i,j=1,n},\quad 
\tr(\bta ) := \sum_{i=1}^n \tau_{ii},\quad
\bta ^\rd := \bta  - \frac{1}{n}\,\tr(\bta )\,\bI,\qan 
\bta :\bze := \sum_{i,j=1}^n \tau_{ij}\,\zeta_{ij}\,.
\end{equation*}
Furthermore, for each $t \in [1,+\infty)$ we introduce the Banach space
\begin{equation*}\label{Hdiv-tensorial-def}
\bbH(\bdiv_t;\Omega) := \Big\{\bta \in\bbL^2(\Omega):\quad \bdiv(\bta )\in\bL^ t(\Omega)\Big\}\,,
\end{equation*}
equipped with the natural norm 
\begin{equation*}
    \|\bta \|_{\bdiv_t;\Omega} \,:=\, \|\bta \|_{0,\Omega} + \|\bdiv(\bta )\|_{0,t;\Omega} \quad \forall\,\bta \, \in \bbH(\bdiv_t;\Omega)\,.
\end{equation*}
Additionally, we recall that, proceeding as in \cite[eq. (1.43), Section 1.3.4]{g-SPRINGER-2014} (see also \cite[Section 4.1]{cmo2018} 
and \cite[Section 3.1]{cgm-M2AN-2020}), one can prove that for $t \in \left\lbrace \begin{array}{cc}
     (1,+\infty]\,\,\mathrm{in}\,\,\R^2\,, \\[0.5ex]
     [\frac{6}{5},+\infty]\,\,\mathrm{in}\,\,\R^3\,,
\end{array}\right.$ there holds
\begin{equation}\label{ibp-Hdiv-tensorial}
    \pil\bta \bn,\bv\pir = \int_\Omega \Big\{\bta :\nabla \bv + \bv\cdot\bdiv(\bta )\Big\}\,\quad \forall\,(\bta ,\bv) \in \bbH(\bdiv_t;\Omega) \,\times\,\bH^1(\Omega)\,,
\end{equation}
where $\pil\cdot,\cdot\pir$ denotes in \eqref{ibp-Hdiv-tensorial} the 
duality pairing between $\bH^{-1/2}(\Gamma)$ and $\bH^{1/2}(\Gamma)$. 

\section{The model problem}\label{section2}

We consider a porous medium occupying the region $\Omega$, and assume that a viscous fluid governed 
by the Brinkman-Forchheimer equations flows through it, so that the sought variables are its pressure 
$p$ and velocity $\bu$. In addition, we let $\phi$ be the concentration of a chemical component 
transported by the fluid, which is advected and diffused in $\Omega$ according to the corresponding 
physical principle. Alternatively, $\phi$ could represent the temperature of the fluid, among several 
other possibilities. In this way, the coupled model of interest is given by the following system 
of partial differential equations:
\begin{equation}\label{BF-transport}
\begin{array}{rcll}
\ds
-\,\mu \, \Delta\bu \,+\, \tD\,\bu \,+\, \tF\,|\bu|^{\rho-2}\,\bu \,+\, \nabla p &=& \phi\,\f & \qin \Omega\,, \\[1.5ex]
\ds
\div(\bu) &=& 0 & \qin \Omega\,, \\[1.5ex]
\ds
\div\big( \vartheta(|\nabla\phi|) \, \nabla\phi - \phi\,\bu \,-\,f(\phi) \,\mathbf g\big) &=& g & \qin \Omega\,, \\[1.5ex]
\bu \,=\, \bu_\rD \qan \phi &=&\phi_\rD & \qon \Gamma\,,
\end{array}
\end{equation}
where $\mu$ is the constant viscosity of the fluid, $\tD, \, \tF > 0$ are the Darcy 
and Forchheimer coefficients, respectively, $\rho$ is a given number in $[3,4]$, 
$\vartheta : \R^+ \to \R^+$ is a nonlinear diffusivity function, $f$ is a nonlinear 
flux acting in the direction of $\g$, which, in turn, is a constant vector pointing 
in the direction of gravity, $\f$ and $g$ are given source functions, and $\bu_\rD$ and 
$\phi_\rD$ are Dirichlet data for $\bu$ and $\phi$, respectively. Regarding $\vartheta$, 
we assume that there exist constants $\vartheta_1, \, \vartheta_2 > 0$ such that
\begin{equation}\label{assumption-mu}
\vartheta_1 \le \vartheta(t) \le \vartheta_2 \qan \vartheta_1 \le \vartheta(t) + t\,\vartheta'(t) \le \vartheta_2
\qquad\forall\, t\in \R^+\,.
\end{equation}
In addition, $f$ is required to be bounded and Lipschitz-continuous, which means that
there exist constants $f_1, \, f_2, \, \L_f > 0$ such that
\begin{equation}\label{assumption-f}
f_1 \le f(t) \le f_2 \qan |f(t) - f(s)| \,\le\, \L_f\,|s-t| \qquad \forall\,s, \, t \in \R^+\,.
\end{equation}

Now, due to the incompressibility of the fluid (cf. second row of \eqref{BF-transport}),
$\bu_\rD$ must formally satisfy the compatibility condition
\begin{equation}\label{eq:compatibility-condition}
\int_{\Gamma} \bu_\rD\cdot\bn \,=\, 0 \,.
\end{equation}
On the other hand, for the uniqueness of $p$ we look for this unknown in the space
\begin{equation*}
\L^2_0(\Omega) \,:=\, \left\{ q\in \L^2(\Omega) :\quad \int_{\Omega} q = 0 \right\} \,.
\end{equation*}

Next, in order to derive a mixed-primal formulation for \eqref{BF-transport}, 
we first define as an auxiliary unknown the fluid pseudostress 
\begin{equation}\label{def-bsi}
\bsi \,:=\, \mu\, \nabla\bu \,-\, p\,\bI\,,
\end{equation}
so that the first row of \eqref{BF-transport} becomes
\begin{equation}\label{eq-BF-1}
-\,\bdiv(\bsi) \,+\, \tD\,\bu \,+\, \tF\,|\bu|^{\rho-2}\,\bu \,=\, \phi\,\f \,.
\end{equation}
Thus, taking matrix trace along with the fact that $\tr(\nabla\bu) = \div(\bu) = 0$,
and then applying the deviatoric operator, we deduce from \eqref{def-bsi} that
\begin{equation}\label{p-bsi}
p\,=\, -\dfrac{1}{n} \,\tr(\bsi) \qan \dfrac{1}{\mu}\,\bsi^\rd\,=\, \nabla \bu \,,
\end{equation}
which are equivalent to the pair of equations formed by the incompressibility condition
and \eqref{def-bsi}. 
Hence, eliminating the unknown $p$, and computing it afterwards according to the identity
provided in \eqref{p-bsi}, the original system \eqref{BF-transport} can be stated, equivalently,
as: Find $\bsi$, $\bu$, and $\phi$ in suitable spaces to be indicated below, such that
\begin{equation}\label{BF-transport-equivalent}
\begin{array}{rcll}
\ds
\dfrac{1}{\mu}\,\bsi^\rd &=& \nabla \bu & \qin \Omega\,, \\[3ex]
\ds
-\,\bdiv(\bsi) \,+\, \tD\,\bu \,+\, \tF\,|\bu|^{\rho-2}\,\bu &=& \phi\,\f & \qin \Omega\,, \\[1.5ex]
\ds
\div\big(\vartheta(|\nabla\phi|) \, \nabla\phi - \phi\,\bu \,-\,f(\phi) \,\mathbf g\big) &=& g & \qin \Omega\,, \\[2ex]
\ds
\bu \,=\, \bu_\rD \qan \phi &=&\phi_\rD & \qon \Gamma\,, \\[2ex]
\ds
\int_\Omega \tr(\bsi) &=& 0 \,. & 
\end{array}
\end{equation}


\section{The variational formulation}\label{section3}

{\color{blue}
\noindent
Find $(\bsi,\bu) \in \H \times \Q$ and $(\phi,\xi) \in \X \times \Y$ such that
\begin{equation}\label{mixed-formulation}
\begin{array}{lcll}
\ds
\ba(\bsi,\bta) + \bb(\bta,\bu)&=&\bF(\bta)&
\quad\forall\,\bta \in \H\,,\\[2ex]
\ds
\bb(\bsi,\bv)\,-\, \bc_\bu(\bu,\bv) &=& \bG_\phi(\bv)& \quad\forall\,\bv\in \Q\,,\\[2ex]
[\A_\bu(\phi),\varphi] + \B(\varphi,\xi) &=& \F_\phi(\varphi) & \quad\forall\,\varphi \in \X\,,\\[2ex]
\B(\phi,\eta) &=& \G(\eta) & \quad\forall\,\eta\in \Y\,,
\end{array}
\end{equation}
where the bilinear forms $\ba : \H \times \H \to \R$, $\bb : \H \times \Q \to \R$, and
$\bc_\bz : \Q \times\Q \to \R$, and the nonlinear operator $\A_\bz : \X \to \X'$, for each $\bz \in \Q$, 
and the bilinear form $\B : \X \times \Y \to \R$, and the linear functionals $\bF : \H \to \R$, $\bG_\psi : \Q \to \R$
and $\F_\psi : \X\to \R$, for each $\psi \in \X$, and $\G : \Y \to \R$, are defined as
\begin{equation}\label{def-ba}
\ba(\bze,\bta) \,:=\, \dfrac1\mu \int_\Omega \bze^\rd : \bta^\rd \qquad\forall\,\bze, \, \bta \in \H \,,
\end{equation}
\begin{equation}\label{def-bb}
\bb(\bta,\bv) \,:=\, \int_\Omega \bv \cdot \bdiv(\bta) \qquad\forall\,(\bta,\bv) \in \H \times \Q\,,
\end{equation}
\begin{equation}\label{def-bc-z}
\bc_\bz(\bw,\bv) \,:=\, \tD \int_\Omega\bw\cdot\bv \,+\, \tF \int_\Omega |\bz|^{\rho-2} \bw\cdot\bv
\qquad \forall\,\bw, \, \bv \in \Q\,,
\end{equation}
\begin{equation}\label{def-A-z}
[\A_\bz(\psi),\varphi] \,:=\, \int_\Omega 
\vartheta(|\nabla\psi|) \, \nabla\psi \cdot \nabla\varphi - \int_\Omega \psi\,\bz \cdot \nabla\varphi
\qquad\forall\, \psi, \, \varphi \in \X\,, 
\end{equation}
\begin{equation}\label{def-B}
\B(\varphi,\eta) \,:=\, \langle\eta,\varphi\rangle \qquad\forall\,(\varphi,\eta) \in \X \times \Y\,,
\end{equation}
\begin{equation}\label{def-bF}
\bF(\bta) \,:=\, \langle \bta\,\bn,\bu_\rD\rangle \qquad\forall\,\bta \in \H\,,
\end{equation}
\begin{equation}\label{def-bG-psi}
\bG_\psi(\bv) \,:=\, - \int_\Omega \psi \,\f \cdot \bv \qquad\forall\, \bv \in \Q\,,
\end{equation}
\begin{equation}\label{def-F-psi}
\F_\psi(\varphi) \,:=\, \int_\Omega f(\psi)\,\g \cdot \nabla\varphi \,-\, \int_\Omega g\,\varphi 
\qquad\forall\,\varphi \in \X\,,
\end{equation}
and
\begin{equation}\label{def-G}
\G(\eta)\,:=\, \langle \eta,\phi_\rD\rangle \qquad\forall\, \eta \in \Y\,.
\end{equation}
Note here that $[\,\cdot,\cdot\,]$ represents  the duality pairing between $\X'$ and $\X$.
In turn, in the definitions of $\B$ (cf. \eqref{def-B}) and $\G$ (cf. \eqref{def-G}),
$\langle\,\cdot,\cdot\,\rangle$ stands for the duality pairing between $\H^{-1/2}(\Gamma)$
and $\H^{1/2}(\Gamma)$, whereas in the definition of $\bF$ (cf. \eqref{def-bF}) $\langle\,\cdot,\cdot\,\rangle$  
denotes the duality pairing between $\bH^{-1/2}(\Gamma)$ and $\bH^{1/2}(\Gamma)$.}





\section{The Galerkin scheme}\label{sec:Galerkin-scheme}


\section{A priori error analysis}\label{sec:a-priori-error-analysis}


\section{Numerical results}\label{sec:numerical-results}


\begin{thebibliography}{99}

%\bibitem{bcgh2020}
%{\sc G.A. Benavides, S. Caucao, G.N. Gatica and A.A. Hopper},
%{\it A Banach spaces-based analysis of a new mixed-primal finite element method for a coupled flow-transport problem}. 
%Comput. Methods Appl. Mech. Engrg. 371 (2020), 113285, 29 pp.
%
%\bibitem{bcgh2022}
%{\sc G.A. Benavides, S. Caucao, G.N. Gatica and A.A. Hopper},
%{\it A new non-augmented and momentum-conserving fully-mixed finite element method for a coupled flow-transport problem}. 
%Calcolo 59 (2022), no. 1, Paper No. 6, 44 pp.
%
%\bibitem{Brezzi-Fortin}
%{\sc F. Brezzi and M. Fortin},
%{\it Mixed and Hybrid Finite Element Methods}.
%Springer Series in Computational Mathematics, 15. Springer-Verlag, New York, 1991.

%\bibitem{cgo2021}
%{\sc J. Cama\~no, C. Garc\'ia and R. Oyarz\'ua},
%{\it Analysis of a momentum conservative mixed-FEM for the stationary Navier-Stokes problem}.
%Numer. Methods Partial Differential Equations 37 (2021), no. 5, 2895--2923. 
%
%\bibitem{cgot2016}
%{\sc J. Cama\~no, G.N. Gatica, R. Oyarz\'ua and G. Tierra},
%{\it An augmented mixed finite element method for the Navier-Stokes equations with variable viscosity}.
%SIAM J. Numer. Anal. 54 (2016), no. 2, 1069--1092.

\bibitem{cmo2018}
{\sc J. Cama\~no, C. Mu\~noz and R. Oyarz\'ua},
{\it Numerical analysis of a dual-mixed problem in non-standard Banach spaces}.
Electron. Trans. Numer. Anal. 48 (2018), 114--130.

%{\color{blue}
%\bibitem{ccgi-2023}
%{\sc S. Caucao, E. Colmenares, G.N. Gatica and C. Inzunza},
%{\it A Banach spaces-based fully-mixed finite element method for the stationary chemotaxis-Navier--Stokes problem}.
%Comp. Math. Appl. 145 (2023), 65--89. 
%}

%\bibitem{ce2023-pp}
%{\sc S. Caucao and J. Esparza},
%{\it An augmented mixed FEM for the convective Brinkman--Forchheimer problem: a priori and a posteriori error analysis}.
%arXiv:2209.02894 [math.NA], (2023).

%\bibitem{cgos2020}
%{\sc S. Caucao, G.N. Gatica, R. Oyarz\'ua and N. S\'anchez},
%{\it A fully-mixed formulation for the steady double-diffusive convection system based upon Brinkman--Forchheimer equations}.
%J. Sci. Comput. 85 (2020), no. 2, Paper No. 44, 37 pp. 

%\bibitem{cov2020}
%{\sc S. Caucao, R. Oyarz\'ua and S. Villa-Fuentes},
%{\it A new mixed-FEM for steady-state natural convection models allowing conservation of momentum and thermal energy}.
%Calcolo 57 (2020), no. 4, Paper No. 36, 39 pp. 
%
%\bibitem{covy2022}
%{\sc S. Caucao, R. Oyarz\'ua, S. Villa-Fuentes and I. Yotov},
%{\it A three-field Banach spaces-based mixed formulation for the unsteady Brinkman--Forchheimer equations}.
%Comput. Methods Appl. Mech. Engrg. 394 (2022), Paper No. 114895, 32 pp. 
%
%\bibitem{cy2021}
%{\sc S. Caucao and I. Yotov,}
%{\it A Banach space mixed formulation for the unsteady Brinkman-Forchheimer equations}. 
%IMA J. Numer. Anal. 41 (2021), no. 4, 2708--2743.

%\bibitem{cku2005}
%{\sc A.O. Celebi, V.K. Kalantarov and D. Ugurlu},
%{\it Continuous dependence for the convective Brinkman--Forchheimer equations}.
%Appl. Anal. 84 (2005), no. 9, 877--888.

%\bibitem{Ciarlet}
%{\sc P.G. Ciarlet},
%{\it Linear and Nonlinear Functional Analysis with Applications}.
%Society for Industrial and Applied Mathematics, Philadelphia, PA, 2013.

\bibitem{cgm-M2AN-2020}
{\sc E. Colmenares, G.N. Gatica and S. Moraga},
{\it A Banach spaces-based analysis of a new fully-mixed finite element method for the Boussinesq problem}.
ESAIM Math. Model. Numer. Anal. 54 (2020), no. 5, 1525--1568.

%\bibitem{cn-CAMWA-2016}
%{\sc E. Colmenares and M. Neilan}, 
%{\it Dual-mixed finite element methods for the stationary Boussinesq problem}. 
%Comp. Math. Appl. 72 (2016), no. 7, 1828--1850.

%{\color{blue}
%\bibitem{crrb2021}
%{\sc P.-H. Cocquet, M. Rakotobe, D. Ramalingom, and A. Bastide}, 
%{\it Error analysis for the finite element approximation of the Darcy--Brinkman--Forchheimer model for porous media with mixed boundary conditions}.
%J. Comput. Appl. Math. 381 (2021), 113008, 24 pp.
%}

%\bibitem{cg-CAMWA-2022}
%{\sc C.I. Correa and G.N. Gatica},
%{\it On the continuous and discrete well-posedness of perturbed saddle-point formulations in Banach spaces}.
%Comput. Math. Appl. 117 (2022), 14--23.
%
%\bibitem{cgr-M2AN-2023}
%{\sc C.I. Correa, G.N. Gatica and R. Ruiz-Baier},
%{\it New mixed finite element methods for the coupled Stokes and Poisson-Nernst-Planck equations in Banach spaces}.
%ESAIM Math. Model. Numer. Anal., https://doi.org/10.1051/m2an/2023024.

%\bibitem{Ern-Guermond}
%{\sc A. Ern and J.-L. Guermond},
%{\it Theory and Practice of Finite Elements}.
%Applied Mathematical Sciences, 159. Springer-Verlag, New York, 2004.

\bibitem{g-SPRINGER-2014}
{\sc G.N. Gatica},
{\it A Simple Introduction to the Mixed Finite Element Method. Theory and Applications}.
SpringerBriefs in Mathematics. Springer, Cham, 2014.

%\bibitem{ggm2014}
%{\sc G.N. Gatica, L.F. Gatica, and A. M\'arquez},
%{\it Analysis of a pseudostress-based mixed finite element method for the Brinkman model of porous media flow}.
%Numer. Math. 126 (2014), no. 4, 635--677.

%\bibitem{gnr2023}
%{\sc G.N. Gatica, N. N\'u\~nez and R. Ruiz-Baier},
%{\it New non-augmented mixed finite element methods for the Navier--Stokes--Brinkman equations using Banach spaces}.
%J. Numer. Math., https://doi.org/10.1515/jnma-2022-0073.
%
%\bibitem{gos2018}
%{\sc L.F. Gatica, R. Oyarz\'ua and N. S\'anchez},
%{\it A priori and a posteriori error analysis of an augmented mixed-FEM for the Navier-Stokes-Brinkman problem}.
%Comput. Math. Appl. 75 (2018), no. 7, 2420--2444. 
%
%{\color{blue}	
%\bibitem{Girault-Raviart}
%{\sc V. Girault and P.A. Raviart},
%{\it Finite Element Methods for Navier--Stokes Equations. Theory and Algorithms}.
%Springer Series in Computational Mathematics, 5. Springer-Verlag, Berlin, 1986.}

%\bibitem{gm1975}
%{\sc R. Glowinski and A. Marrocco},
%{\it Sur l'approximation, par éléments finis d'ordre un, et la résolution, par pénalisations-dualité d'une classe de probl\'emes de Dirichlet non lineaires}.
%R.A.I.R.O. tome 9, no 2 (1975), p. 41-76.
%	
%\bibitem{Hecht2012}
%{\sc F. Hecht},
%{\it New development in FreeFem++}.
%J. Numer. Math. 20 (2012), 251-–265.

%\bibitem{ll2019}
%{\sc D. Liu and K. Li},
%{\it Mixed finite element for two-dimensional incompressible convective Brinkman-Forchheimer equations}.
%Appl. Math. Mech. (English Ed.) 40 (2019), no. 6, 889--910. 

%{\color{blue}
%\bibitem{mr2009}
%{\sc V. Maz'ya and J. Rossmann},
%{\it Mixed boundary value problems for the stationary Navier--Stokes system in polyhedral domains}.
%Arch. Rational Mech. Anal. 194 (2009), 669--712.
%}

%{\color{blue}
%\bibitem{sw2017}
%{\sc P. Skrzypacz and D. Wei}, 
%{\it Solvability of the Brinkman--Forchheimer--Darcy equation}.
%J. Appl. Math. 2017, Art. ID 7305230, 10 pp.
%}
%
%{\color{blue}
%\bibitem{vp2017}
%{\sc C. Varsakelis and M.V. Papalexandris}, 
%{\it On the well-posedness of the Darcy--Brinkman--Forchheimer equations for coupled porous media-clear fluid flow}.
%Nonlinearity 30 (2017), no. 4, 1449--1464.
%}

%\bibitem{y2023}
%{\sc H. Yu},
%{\it Axisymmetric solutions to the convective Brinkman-Forchheimer equations}.
%J. Math. Anal. Appl. 520 (2023), no. 2, Paper No. 126892, 12 pp.
%
%\bibitem{zy2012}
%{\sc C. Zhao and Y. You},
%{\it Approximation of the incompressible convective Brinkman--Forchheimer equations}.
%J. Evol. Equ. 12 (2012), no. 4, 767--788.
	
\end{thebibliography}
\end{document}




%*********************************************************************************************