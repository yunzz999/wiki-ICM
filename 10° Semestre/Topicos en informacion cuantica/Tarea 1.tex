\documentclass[10pt,a4paper]{report}
\usepackage[utf8]{inputenc}
\usepackage[spanish]{babel}
\usepackage[T1]{fontenc}
\usepackage{amsmath}
\usepackage{amsfonts}
\usepackage{amssymb}
\usepackage{braket}
\usepackage{amsthm}
\usepackage{ dsfont }
\usepackage{stmaryrd}
\usepackage{stix}
\usepackage[mathscr]{euscript}
% Bibliografia
\usepackage[backend=biber]{biblatex}
\bibliography{Referencias.bib}
% Cajas
\usepackage{tcolorbox}
\usepackage{tikz}
% Comandos
\newtheorem{teo}{Teorema}
\newtheorem{prop}{Proposición}
\newtheorem{pro}{Problema}
\newcommand{\autor}{\textbf{Brayan Sandoval}}
\newcommand{\asignatura}{\textbf{Tópicos En Información Cuántica}}
\newcommand{\tarea}{\textbf{Tarea 1}}
\newcommand{\fecha}{\textbf{\today}}
\newcommand{\bs}{\boldsymbol}
\newcommand{\dx}{\textup{d}x}
\newcommand{\dy}{\textup{d}y}
\newcommand{\dt}{\textup{d}t}
\newcommand{\ds}{\textup{d}s}
\newcommand{\dS}{\textup{d}S}
\newcommand{\DG}{\textup{DG}}
\newcommand{\Th}{\mathscr{T}_{h}}
\newcommand{\Hdiv}{H\left(\textup{div};\Omega\right)}
\newcommand{\Hdivt}{H(\textup{div};\mathscr{T}_{h})}
\providecommand{\tr}[1]{\textup{tr}\left(#1\right)}
\providecommand{\Dt}[1]{\frac{\textup{d} #1}{\textup{d}t}}
\providecommand{\Dx}[1]{\frac{\textup{d} #1}{\textup{d}x}}
\providecommand{\abs}[1]{\left\lvert#1\right\rvert}
\providecommand{\norm}[1]{\left\lVert#1\right\rVert}
\providecommand{\salto}[1]{\left\llbracket#1\right\rrbracket}
\providecommand{\prom}[1]{\left \{\!\left \{#1\right \}\!\right \}}
\providecommand{\PI}[2]{\left\langle #1,#2  \right\rangle}
\providecommand{\Pii}[2]{\left( #1,#2  \right)}
\renewcommand{\theequation}{\roman{equation}}

%texto
\usepackage{lipsum} % Genera texto aleatorio
\renewcommand*{\familydefault}{\sfdefault} % Letra mas bonita
% Figuras
\usepackage{graphicx}
% Geometría
\usepackage[left= 2 cm, right = 2 cm, top = 2 cm, bottom = 2 cm]{geometry}
\usepackage{lastpage}
% Color
\usepackage{xcolor}
\definecolor{azul}{RGB}{10,10,115}
\definecolor{amarillo}{RGB}{255,204,0}
\definecolor{rojo}{RGB}{247,0,30}
% Encabezado
\usepackage{fancyhdr}
\pagestyle{fancy}
\renewcommand{\headrulewidth}{4pt} %Aumentar grosor linea encabezado
\let\oldheadrule\headrule
\renewcommand{\headrule}{\color{azul}\oldheadrule}
\renewcommand{\footrulewidth}{4pt} %Aumentar grosor linea pie de pagina
\let\oldfootrule\footrule
\renewcommand{\footrule}{\color{azul}\oldfootrule}
\rhead{\color{azul}\autor}
\chead{\color{azul}\tarea}
\lhead{\color{azul}\asignatura}
\rfoot{\color{azul} \textbf{Pág. \thepage\ - \pageref{LastPage}}}
\cfoot{}
\lfoot{\color{azul}\fecha}
% Titulo
\title{\color{azul}\textbf{Tópicos En Información Cuántica}\\
	\textbf{Tarea 1}}
\author{\color{azul}\autor}
\date{\color{azul}\fecha}
\begin{document}
	\begin{pro}
		\textit{Entrelazamiento de los estados de Werner.}\\
		El estado de Werner de $2$ qubits está definido por 
		\begin{align*}
			\rho_{w} = p\ket{\varPhi_{-}}\bra{\varPhi_{-}} + \frac{1-p}{4}\mathds{1},
		\end{align*}
	    donde $p\in[0,1]$ y $\ket{\varPhi_{-}} = \frac{1}{\sqrt{2}}\big(\ket{01}-\ket{10}\big)$ es un estado de Bell.
	    \begin{itemize}
	    	\item[$1.$] Muestre que $\rho_{w}$ es un operador densidad y calcule su pureza $\tr{\rho^{2}_{w}}$.
	    	\item[$2.$] Use el criterio de Peres-Horodecki para hallar una condición necesaria y suficiente sobre el parámetro $p$ para que $\rho_{w}$ sea un estado entrelazado.
	    \end{itemize}
	\end{pro}
    \begin{proof}
    	Sea $\mathcal{H} := \mathbb{C}^{2}$ un espacio de Hilbert con la base $\{\ket{0},\ket{1}\}$ y el espacio producto $\mathcal{H}\otimes\mathcal{H}$ con la base $\{\ket{00},\ket{01},\ket{10},\ket{11}\}$. Para mostrar que $\rho_{w}$ es un operador densidad, se debe probar que $\rho_{w}\in \mathcal{E}_{\mathcal{H}^{2}}$ donde 
    	\begin{align*}
    		\mathcal{E}_{\mathcal{H}^{2}} := \{\rho:\mathcal{H}\otimes\mathcal{H}\longrightarrow\mathcal{H}\otimes\mathcal{H}; \ \rho\geq0, \quad \tr{\rho}=1\}.
    	\end{align*}
        Se puede ver que
    	\begin{itemize}
    		\item[$a)$] $\rho_{w}$ es positivo. En efecto, sea $\ket{\varphi}\in \mathcal{H}\otimes\mathcal{H}$, luego
    		\begin{align*}
    			\bra{\varphi}\rho_{w}\ket{\varphi} &= \bra{\varphi}\left(p\ket{\varPhi_{-}}\bra{\varPhi_{-}} + \frac{1-p}{4}\mathds{1}\right)\ket{\varphi} = p\braket{\varphi|\varPhi_{-}}\braket{\varPhi_{-}|\varphi} + \frac{1-p}{4}\bra{\varphi}\mathds{1}\ket{\varphi}\\
    			&= p|\braket{\varphi|\varPhi_{-}}|^{2} + \frac{1-p}{4}\bra{\varphi}\left(\sum_{i,j=0}^{1}\ket{ij}\bra{ij}\right)\ket{\varphi}\\
    			&= p|\braket{\varphi|\varPhi_{-}}|^{2} + \frac{1-p}{4}\sum_{i,j=0}^{1}\braket{\varphi|ij}\braket{ij|\varphi}\\
    			&= p|\braket{\varphi|\varPhi_{-}}|^{2} + \frac{1-p}{4}\sum_{i,j=0}^{1}|\braket{\varphi|ij}|^{2} \geq 0.
    		\end{align*}
    	    \item[$b)$] $\tr{\rho_{w}} = 1$. En efecto, 
    	    \begin{align*}
    	    	\tr{\rho_{w}} &= \tr{p\ket{\varPhi_{-}}\bra{\varPhi_{-}} + \frac{1-p}{4}\mathds{1}}\\ 
    	    	&= p\ \tr{\ket{\varPhi_{-}}\bra{\varPhi_{-}}} + \frac{1-p}{4}\tr{\mathds{1}}\\ 
    	    	&= p\braket{\varPhi_{-}|\varPhi_{-}} + \frac{1-p}{4}4\\
    	    	&= \frac{p}{2}\big(\braket{01|01} - \braket{01|10} - \braket{10|01} + \braket{10|10}\big) + 1-p\\
    	    	&= \frac{p}{2}2 + 1-p\\
    	    	&= p + 1-p\\
    	    	&= 1.
    	    \end{align*}
    	\end{itemize}
        Y por tanto $\rho_{w}\in \mathcal{E}_{\mathcal{H}^{2}}$, es decir, $\rho_{w}$ es un operador densidad. Para determinar la pureza de $\rho_{w}$ se procede como sigue
        \begin{align*}
        	\rho^{2}_{w} &= \left(p\ket{\varPhi_{-}}\bra{\varPhi_{-}} + \frac{1-p}{4}\mathds{1}\right)\left(p\ket{\varPhi_{-}}\bra{\varPhi_{-}} + \frac{1-p}{4}\mathds{1}\right)\\ 
        	&= p^{2}\braket{\varPhi_{-}|\varPhi_{-}} \ket{\varPhi_{-}}\bra{\varPhi_{-}} + \frac{p(1-p)}{4}\ket{\varPhi_{-}}\bra{\varPhi_{-}} + \frac{p(1-p)}{4}\ket{\varPhi_{-}}\bra{\varPhi_{-}}  + \frac{(1-p)^{2}}{16}\mathds{1}\\
        	&= \frac{p^{2}+p}{2}\ket{\varPhi_{-}}\bra{\varPhi_{-}} + \frac{(1-p)^{2}}{16}\mathds{1},
        \end{align*}
        aplicando traza y usando e hecho que es lineal
        \begin{align*}
        	\tr{\rho^{2}_{w}} &= \frac{p^{2}+p}{2}\tr{\ket{\varPhi_{-}}\bra{\varPhi_{-}}} + \frac{(1-p)^{2}}{16}\tr{\mathds{1}}\\
        	&= \frac{p^{2}+p}{2}\braket{\Phi_{-}|\Phi_{-}} + \frac{(1-p)^{2}}{16}4\\
        	&= \frac{p^{2}+p}{2} + \frac{(1-p)^{2}}{4}\\
        	&= \frac{3p^{2}+1}{4}.
        \end{align*} 
       Por tanto, la pureza de $\rho_{w}$ es $\frac{3p^{2}+1}{4}$.\\~\\
       Por otro lado. Sean $\mathcal{H}_{A}:= \mathcal{H}$ y $\mathcal{H}_{B} := \mathcal{H}$, es inmediato notar que $\textup{dim}(\mathcal{H}_{A}) = n_{A} = 2 = n_{B} =  \textup{dim}(\mathcal{H}_{B}) $, por lo que es valido el criterio de Peres-Horodecki. Esto es 
       \begin{align*}
       	\rho_{W}\ \textup{es separable} \iff \rho^{T_{B}}_{W} \ \textup{es un operador densidad.}
       \end{align*}
       Lo anterior motiva a hallar condiciones sobre $p$ para las cuales $\rho^{T_{B}}_{W}$ sea un operador densidad.
       \begin{itemize}
       	\item[$a)$] Positividad de $\rho^{T}_{W}$. En primer lugar, de un calculo directo se tiene que 
       	\begin{align*}
       		\ket{\varPhi_{-}}\bra{\varPhi_{-}} &= \frac{1}{2}\left(\ket{01}-\ket{10}\right)\left(\bra{01}-\bra{10}\right)\\
       		&= \frac{1}{2} \begin{pmatrix}
       			0 & 0 & 0 & 0\\ 
       			0 & 1 & -1 & 0 \\ 
       			0 & -1 & 1 & 0\\ 
       			0 & 0 & 0 & 0
       		\end{pmatrix},
       	\end{align*}
       	luego,
       	\begin{align*}
       		\rho_{W} = \frac{p}{2} \begin{pmatrix}
       			0 & 0 & 0 & 0\\ 
       			0 & 1 & -1 & 0 \\ 
       			0 & -1 & 1 & 0\\ 
       			0 & 0 & 0 & 0
       		\end{pmatrix} + \frac{1-p}{4} \begin{pmatrix}
       		1 & 0 & 0 & 0\\ 
       		0 & 1 & 0 & 0 \\ 
       		0 & 0 & 1 & 0\\ 
       		0 & 0 & 0 & 1
       	\end{pmatrix} = \begin{pmatrix}
       	\frac{1-p}{4} & 0 & 0 & 0\\ 
       	0 & \frac{1+p}{4} & -\frac{p}{2} & 0 \\ 
       	0 & -\frac{p}{2} & \frac{1+p}{4} & 0\\ 
       	0 & 0 & 0 & \frac{1-p}{4}
       \end{pmatrix}
       	\end{align*}
       con lo cual
       \begin{align*}
       	\rho^{T_{B}}_{W} = \begin{pmatrix}
       		\begin{bmatrix}
       			\frac{1-p}{4} & 0\\ 
       			0 & \frac{1+p}{4}
       		\end{bmatrix}^{T} & \begin{bmatrix}
       			0 & 0\\ 
       			-\frac{p}{2} & 0 
       		\end{bmatrix}^{T}\\~\\ 
       		\begin{bmatrix}
       			0 & -\frac{p}{2}\\ 
       			0 & 0
       		\end{bmatrix}^{T} & \begin{bmatrix}
       			\frac{1+p}{4} & 0 \\ 
       			0 & \frac{1-p}{4}
       		\end{bmatrix}^{T}
       	\end{pmatrix} = \begin{pmatrix}
       	\frac{1-p}{4} & 0 & 0 & -\frac{p}{2}\\ 
       	0 & \frac{1+p}{4} & 0 & 0 \\ 
       	0 & 0 & \frac{1+p}{4} & 0\\ 
       	-\frac{p}{2} & 0 & 0 & \frac{1-p}{4}
       \end{pmatrix}.
       \end{align*}
      Una vez encontrada la matriz representante del operador lineal, se procede a obtener lo valores propios de dicha matriz, los cuales se pueden calcular encontrando las raíces del polinomio característico asociado a la matriz, en este caso el polinomio característico viene dado por  
      \begin{align*}
      	p_{\rho^{T_{B}}_{W}}(\lambda) &= \textup{det}\left(\rho^{T_{B}}_{W} - \lambda\mathds{1}\right)\\ &= \begin{vmatrix}
      		\frac{1-p}{4}-\lambda & 0 & 0 & -\frac{p}{2}\\ 
      		0 & \frac{1+p}{4}-\lambda & 0 & 0 \\ 
      		0 & 0 & \frac{1+p}{4}-\lambda & 0\\ 
      		-\frac{p}{2} & 0 & 0 & \frac{1-p}{4}-\lambda
      	\end{vmatrix}\\
      &= \left(\frac{1-p}{4}-\lambda\right)\begin{vmatrix}
      	\frac{1+p}{4}-\lambda & 0 & 0 \\ 
      	0 & \frac{1+p}{4}-\lambda & 0\\
      	0 & 0 & \frac{1-p}{4}-\lambda
      \end{vmatrix} + \frac{p}{2}\begin{vmatrix}
      0 & 0 & -\frac{p}{2} \\ 
      \frac{1+p}{4}-\lambda & 0 & 0\\
      0 & \frac{1+p}{4}-\lambda & 0
  \end{vmatrix}\\
      &= \left(\frac{1-p}{4}-\lambda\right)^{2}\left(\frac{1+p}{4}-\lambda\right)^{2} - \frac{p^{2}}{4}\left(\frac{1+p}{4}-\lambda\right)^{2}\\
      &= \left(\frac{1+p}{4}-\lambda\right)^{2}\left[\left(\frac{1-p}{4}-\lambda\right)^{2} - \frac{p^{2}}{4}\right]\\
      &= \left(\frac{1+p}{4}-\lambda\right)^{2}\left[\frac{1-p}{4}-\lambda - \frac{p}{2}\right]\left[\frac{1-p}{4}-\lambda + \frac{p}{2}\right]\\
      &= \left(\frac{1+p}{4}-\lambda\right)^{3}\left(\frac{1-3p}{4}-\lambda\right)
      \end{align*}
      De lo anterior se desprende que 
      \begin{align*}
      	p_{\rho^{T_{B}}_{W}}(\lambda) = 0 \iff \lambda = \frac{1+p}{4} \ \vee \ \lambda = \frac{1-3p}{4}.
      \end{align*} 
      De esta forma 
      \begin{align*}
      	\rho^{T_{B}}_{W} \geq 0 \ &\iff \ \frac{1+p}{4}\geq0\ \wedge\ \frac{1-3p}{4}\geq0\\
      	                          &\iff \ p\geq -1 \ \wedge \ p\leq \frac{1}{3}.  
      \end{align*} 
      Pero como $p\in[0,1]$ se tiene que
      \begin{align*}
      	  \rho^{T_{B}}_{W} \geq 0 \ &\iff \ p\in\left[0,\frac{1}{3}\right]
      \end{align*}
      \item[$b)$] $\textup{tr}\left(\rho^{T_{B}}_{W}\right) = 1$. Usando el hecho que $\textup{tr}\left(\rho_{W}\right) = 1$ y que la transposición parcial preserva la traza, se concluye lo pedido.
       \end{itemize}
      Finalmente, se deduce que 
      \begin{align*}
      	\rho_{W}\ \textup{es separable} \iff \ p\in\left[0,\frac{1}{3}\right].
      \end{align*}
      Y esto muestra que 
      \begin{align*}
      	\rho_{W}\ \textup{es un estado entrelazado} \iff \ p\in \left(\frac{1}{3}, 1 \right].
      \end{align*}
    \end{proof}
    \newpage
    \begin{pro}
    	\textit{La transposición no es un canal cuántico.}\\
    	Muestre que la aplicación transposición en la base canónica $\{\ket{i}\}^{1}_{i=0}$ de $\mathbb{C}^{2}$,
    	\begin{align*}
    		A \mapsto A^{T} := \sum^{1}_{i,j=0} \bra{j}A\ket{i}\ket{i}\bra{j},
    	\end{align*}
        es una aplicación positiva $\mathcal{B}\left(\mathbb{C}^{2}\right)\rightarrow\mathcal{B}\left(\mathbb{C}^{2}\right)$, pero no es un canal cuántico.
    \end{pro}
    \begin{proof}
    	En primer lugar, es fácil probar que $A$ y $A^{T}$ tienen los mismos valores propios. En efecto 
    	\begin{align*}
    		p_{A^{T}}(\lambda) = \textup{det}\left(A^{T}-\lambda\mathds{1}\right) = \textup{det}\left(\left(A-\lambda\mathds{1}\right)^{T}\right) = \textup{det}\left(A-\lambda\mathds{1}\right) = p_{A}(\lambda).
    	\end{align*}
        Como ambos operadores tienen el mismo polinomio característico se deduce los valores propios de $A$ y $A^{T}$ coinciden. En otras palabras, dado $A$ positivo, se tendrá que el operador $A^{T}$ también es positivo, lo cual equivale a decir, la aplicación transposición es positiva.\\~\\
        Por otro lado, se puede observar que la aplicación transposición no es un canal cuántico ya que no es $2-$positiva. En efecto, considere el operador densidad $\rho_{W}$ definido en el \textbf{Problema 1}, luego 
        \begin{align*}
        	T\otimes\mathds{1}_{\mathbb{C}^{2}}\left(\rho_{W}\right) = \begin{pmatrix}
        		\begin{bmatrix}
        			\frac{1-p}{4} & 0\\ 
        			0 & \frac{1+p}{4}
        		\end{bmatrix}^{T} & \begin{bmatrix}
        			0 & 0\\ 
        			-\frac{p}{2} & 0 
        		\end{bmatrix}^{T}\\~\\ 
        		\begin{bmatrix}
        			0 & -\frac{p}{2}\\ 
        			0 & 0
        		\end{bmatrix}^{T} & \begin{bmatrix}
        			\frac{1+p}{4} & 0 \\ 
        			0 & \frac{1-p}{4}
        		\end{bmatrix}^{T}
        	\end{pmatrix} = \rho^{T_{B}}_{W}.
        \end{align*} 
       Por lo hecho en el \textbf{Problema 1}, se tiene para $p\in \left( \frac{1}{3},1 \right]$ que $\rho^{T_{B}}_{W}$ no es positivo, entonces, $T\otimes\mathds{1}_{\mathbb{C}^{2}}$ no puede ser un operador positivo y por tanto  la aplicación transposición no es $2-$positiva.
    \end{proof}
    \newpage
    \begin{pro}\textit{Propiedad del número de Schmidt.}
    	\begin{enumerate}
    		\item Muestre que el número de Schmidt de un estado puro $\ket{\Psi}$ de un sistema compuesto $AB$ es igual al rango de la matriz densidad reducida $\rho_{A} = \underset{B}{\textup{tr}}\ket{\Psi}\bra{\Psi}$.
    		\item Muestre que si
    		\begin{align*}
    			\ket{\Psi} = \sum_{j=1}^{m}c_{j}\ket{\psi_{j}}\ket{\phi_{j}},
    		\end{align*}
    	    donde $\{\ket{\psi_{j}}\}^{m}_{j=1}$ y $\{\ket{\phi_{j}}\}^{m}_{j=1}$ son familias (no necesariamente ortogonales) de vectores de $\mathcal{H}_{A}$ y $\mathcal{H}_{B}$ y $c_{j}\in \mathbb{C}$, $c_{j}\neq 0$, luego $m$ es igual o mayor al número de Schmidt de $\ket{\psi}$.
    	    \item Suponga que $\ket{\Psi} = c_{1}\ket{\Phi_{1}} + c_{2}\ket{\Phi_{2}}$, donde $\ket{\Phi_{1}}$ y $\ket{\Phi_{2}}$ son estados puros de $AB$ y $c_{1},c_{2}\in \mathbb{C}$, $c_{1}\neq 0$, $c_{2}\neq 0$. Muestre que los números de Schmidt $q_{1},q_{2}$ y $q$ de $\ket{\Phi_{1}}$, $\ket{\Phi_{2}}$ y $\ket{\varPhi}$, respectivamente, satisfacen la desigualdad $q\geq |q_{1}-q_{2}|$.
    	\end{enumerate}
    \end{pro}
    \begin{proof}
    	\begin{enumerate}
    		\item Por teorema, se llega a 
    		\begin{align*}
    			\ket{\Psi} = \sum^{q}_{i=1}\sqrt{\mu_{i}}\ket{\alpha_{i}}_{A}\ket{\beta_{i}}_{B},
    		\end{align*}
    	    donde $q\leq \min\{n_{A},n_{B}\}$, $\mu_{i}>0$ para todo $i\in\{1,...,q\}$ y $\{\ket{\alpha_{i}}_{A}\}^{q}_{i=1}$, $\{\ket{\beta_{i}}_{B}\}^{q}_{i=1}$ son familias $||\perp||$ de vectores de $\mathcal{H}_{A}$ y $\mathcal{H}_{B}$. Ademas, los $\mu_{i}$, $\ket{\alpha_{i}}$ y $\ket{\beta_{i}}$ son los valores y vectores propios no nulos de  $\rho_{A} = \underset{B}{\textup{tr}}\ket{\Psi}\bra{\Psi}$ y $\rho_{B} = \underset{A}{\textup{tr}}\ket{\Psi}\bra{\Psi}$. Luego usando la descomposición de Schmidt en la matriz de densidad reducida $\rho_{A}$
    	    \begin{align*}
    	    	\rho_{A} &= \underset{B}{\textup{tr}}\left[\left(\sum^{q}_{i=1}\sqrt{\mu_{i}}\ket{\alpha_{i}}_{A}\ket{\beta_{i}}_{B}\right)\left(\sum^{q}_{j=1}\sqrt{\mu_{j}}\bra{\alpha_{j}}_{A}\bra{\beta_{j}}_{B}\right)\right]\\
    	    	&= \underset{B}{\textup{tr}}\left[\sum^{q}_{i,j=1}\sqrt{\mu_{i}\mu_{j}}\ket{\alpha_{i}}_{A}\bra{\alpha_{j}}_{A}\ket{\beta_{i}}_{B}\bra{\beta_{j}}_{B}\right]\\
    	    	&= \sum^{q}_{i,j=1}\sqrt{\mu_{i}\mu_{j}}\ \underset{B}{\textup{tr}}\left(\ket{\alpha_{i}}_{A}\bra{\alpha_{j}}_{A}\ket{\beta_{i}}_{B}\bra{\beta_{j}}_{B}\right)\\
    	    	&= \sum^{q}_{i,j=1}\sqrt{\mu_{i}\mu_{j}}\ket{\alpha_{i}}_{A}\bra{\alpha_{j}}_{A} \textup{tr}\ket{\beta_{i}}_{B}\bra{\beta_{j}}_{B}\\
    	    	&= \sum^{q}_{i,j=1}\sqrt{\mu_{i}\mu_{j}}\ket{\alpha_{i}}_{A}\bra{\alpha_{j}}_{A}\braket{\beta_{i}|\beta_{j}}_{B}\\
    	    	&= \sum^{q}_{i=1}\mu_{i}\ket{\alpha_{i}}_{A}\bra{\alpha_{i}}_{A}
    	    \end{align*}
            Luego, por teorema de las dimensiones 
            \begin{align*}
            	n_{A} = r(\rho_{A}) + n(\rho_{A}) 
            \end{align*}
            por lo que, demostrar $r\left(\rho_{A}\right) = q$, es equivalente a probar que 
            $n\left(\rho_{A}\right) = n_{A}-q$. Usando que $\{\ket{\alpha_{i}}\}^{q}_{i=1}$ es una familia $||\perp||$, se deduce que es un conjunto linealmente independiente y por tanto $\{\ket{\alpha_{i}}\}^{q}_{i=1}$ es una base de $S = \textup{span}\left(\{\ket{\alpha_{q}},\dots,\ket{\alpha_{q}}\}\right)$. Así, por teorema de descomposición ortogonal
            \begin{align*}
            	\mathcal{H}_{A} = S\oplus S^{\perp},
            \end{align*}
            de aquí, se deduce que $\textup{dim}\left(S^{\perp}\right) = n_{A}-q$. Luego, dado $\ket{\varPhi}\in \textup{Ker}\left(\rho_{A}\right)$, entonces
            \begin{align*}
            	0 &= \rho_{A}\ket{\varPhi}\\ 
            	  &= \sum^{q}_{i=1}\mu_{i}\ket{\alpha_{i}}_{A}\braket{\alpha_{i}|\varPhi}\\ 
            	  &= \sum^{q}_{i=1}\mu_{i}\ket{\alpha_{i}}_{A}\braket{\alpha_{i}|\textup{Proj}_{S}\left(\ket{\varPhi}\right) + \textup{Proj}_{S^{\perp}}\left(\ket{\varPhi}\right)} \}\\ 
            	  &= \sum^{q}_{i=1}\mu_{i}\braket{\alpha_{i}|\textup{Proj}_{S}\left(\ket{\varPhi}\right)}\ket{\alpha_{i}}_{A},
            \end{align*} 
            dado que $\{\alpha_{i}\}^{q}_{i=1}$ es linealmente independiente, se tiene que 
            \begin{align*}
            	\braket{\alpha_{i}|\textup{Proj}_{S}\left(\ket{\varPhi}\right)} = 0,\quad \forall i\in\{1,\dots,q\}
            \end{align*}
            y como $\textup{Proj}_{S}\left(\ket{\varPhi}\right)\notin S^{\perp}$, se tiene que $\textup{Proj}_{S}\left(\ket{\varPhi}\right) = 0_{A}$. Esto implica que $\varPhi\in S^{\perp}$ y por tanto $\textup{Ker}\left(\rho_{A}\right)\subset S^{\perp}$. Recíprocamente, dado $\ket{\varPhi}\in S^{\perp}$, luego
            \begin{align*}
            	\rho_{A}\ket{\varPhi} =  \sum^{q}_{i=1}\mu_{i}\ket{\alpha_{i}}_{A}\braket{\alpha_{i}|\varPhi}_{A} = 0
            \end{align*}
            Así, $S^{\perp}\subset\textup{Ker}\left(\rho_{A}\right)$ y por tanto $S^{\perp}=\textup{Ker}\left(\rho_{A}\right)$. Lo anterior implica que $n\left(\rho_{A}\right) = n_{A}-q$, lo cual es equivalente a decir que $r\left(\rho_{A}\right) = q$.
            \item Usando la descomposición descrita en el enunciado
            \begin{align*}
            	\rho_{A} &= \underset{B}{\textup{tr}}\left[\left(\sum^{m}_{i=1}c_{i}\ket{\psi_{i}}\ket{\phi_{i}}\right)\left(\sum^{m}_{j=1}\bar{c}_{j}\bra{\psi_{j}}\bra{\phi_{j}}\right)\right]\\
            	&= \underset{B}{\textup{tr}}\left[\sum^{m}_{i,j=1}c_{i}\bar{c}_{j}\ket{\psi_{i}}\bra{\psi_{j}}\ket{\phi_{i}}\bra{\phi_{j}}\right]\\
            	&= \sum^{m}_{i,j=1}c_{i}\bar{c}_{j}\ \underset{B}{\textup{tr}}\left(\ket{\psi_{i}}\bra{\psi_{j}}\ket{\phi_{i}}\bra{\phi_{j}}\right)\\
            	&= \sum^{m}_{i,j=1}c_{i}\bar{c}_{j}\ket{\psi_{i}}\bra{\psi_{j}} \textup{tr}\ket{\phi_{i}}\bra{\phi_{j}}\\
            	&= \sum^{m}_{i,j=1}c_{i}\bar{c}_{j}\braket{\phi_{i}|\phi_{j}}\ket{\psi_{i}}\bra{\psi_{j}}.
            \end{align*}
            Usando que la familia $\{\ket{\Psi_{i}}\}^{m}_{i=1}$ no es necesariamente ortogonal y por tanto, no necesariamente linealmente independiente se deduce que
            \begin{align*}
            	r\left(\rho_{A}\right) \leq m.
            \end{align*}
            Además, usando el hecho que $\ket{\Psi}$ es un estado puro, se deduce que $r\left(\rho_{A}\right) = q$. Por tanto
            \begin{align*}
            	q\leq m.
            \end{align*}
            \item En primer lugar, si $q_{1}=q_{2}$, se tiene de forma inmediata la propiedad. Por otro lado si $q_{1}>q_{2}$, notar que 
            \begin{align*}
            	\ket{\Phi_{1}} &= \frac{1}{c_{1}}\ket{\Psi} - \frac{c_{2}}{c_{1}}\ket{\Phi_{2}}\\
            	               &= \frac{1}{c_{1}}\sum^{q}_{j=1}\sqrt{\mu_{j}}\ket{\alpha_{j}}\ket{\beta_{j}} - \frac{c_{2}}{c_{1}}\sum^{q_{2}}_{j=1}\sqrt{\mu_{2,j}}\ket{\alpha_{2,j}}\ket{\beta_{2,j}}\\
            	               &= \sum^{q+q_{2}}_{j=1}a_{j}\ket{\tilde{\alpha}_{j}}\ket{\tilde{\beta}_{j}}.
            \end{align*}
            Por parte 2, $q_{1}\leq q + q_{2}$. Entonces
            \begin{align*}
            	q\geq q_{1}-q_{2}.
            \end{align*}
            Mostrar el caso en que $q_{2}>q_{1}$, es análogo al caso recién demostrado, por tanto
            \begin{align*}
            	q\geq |q_{1}-q_{2}|.
            \end{align*}
    	\end{enumerate}
    \end{proof}
    \newpage
    \begin{pro}
    	\textit{Conjunto de estados puros minimal de un operador densidad}\\
    	Decimos que un conjunto de estados puros $\{\ket{\psi_{i}},p_{i}\}^{r}_{i=1}$ es minimal para el operador densidad $\rho$ si $\rho = \sum_{i}p_{i}\ket{\psi_{i}}\bra{\psi_{i}}$ con $p_{i}>0\quad \forall i= 1,\dots,r$ y el número $r$ de estados del conjunto es igual al rango de $\rho$.
    	\begin{enumerate}
    		\item Muestre que los estados propios normalizados de $\rho$ asociados a valores propios no nulos forman un conjunto minimal para $\rho$.
    		\item Sea $\ket{\psi_{1}}\in\textup{supp}(\rho)$, $||\psi_{1}||=1$, donde $\textup{supp}(\rho)$ es el subespacio imagen de $\mathcal{H}$ por $\rho$. Considere los operadores 
    		\begin{align*}
    			A_{p} = \rho - p\ket{\psi_{1}}\bra{\psi_{1}}
    		\end{align*}
    	    con $0\leq p \leq 1$.
    	    \begin{itemize}
    	    	\item[$(a)$] Muestre que existe un único $p\in ]0,1]$ y un único vector normalizado $\ket{\phi}\in \textup{supp}(\rho)$ módulo un factor de fase tales que $A_{p}\ket{\phi} = 0$. Muestre que estos $p$ y $\ket{\phi}$ están dados por
    	    	\begin{align*}
    	    		p = p_{1} = \bra{\phi_{1}}\rho^{-1}\ket{\psi_{1}}^{-1},\quad \ket{\phi} = c\rho^{-1}\ket{\psi_{1}},
    	    	\end{align*}
        	    donde $c$ es una constante de normalización y se define $\rho^{-1}$ como el inverso del operador $\rho$ restringido a $\textup{supp}(\rho)$.
        	    \item[$(b)$] Muestre que $A_{p}$ es un operador no negativo para todo $p\in[0,p_{1}]$. ¿Cuál es el rango de $A_{p_{1}}$?
        	    \item[$(c)$] Deduzca de las preguntas anteriores que existe un conjunto de estados puros minimal para $\rho$ que contiene $\ket{\psi_{1}}$, en el cual la probabilidad asociada a $\ket{\psi_{1}}$ es igual a $p_{1}$.
    	    \end{itemize}
    	\end{enumerate}
    \end{pro}
    \begin{proof}
    	\begin{enumerate}
    		\item Por descomposición espectral 
    		\begin{align*}
    			\rho = \sum^{d}_{i=1}\lambda_{i}\ket{\varphi_{i}}\bra{\varphi_{i}}
    		\end{align*}
    	    donde $\{\lambda_{i}\}^{d}_{i=1}$ son los valores propios de $\rho$ y $\{\ket{\varphi_{i}}\}^{d}_{i=1}$ es una base $||\perp||$ de vectores propios de $\rho$. Considerando $r$ como el numero de valores propios no nulos de $\rho$, se tiene que
    	    \begin{align*}
    	    	\rho = \sum^{r}_{i=1}\lambda_{i}\ket{\varphi_{i}}\bra{\varphi_{i}},
    	    \end{align*}
            luego, procediendo de forma análoga a la demostración del \textbf{Problema 3.1}, se deduce que 
            \begin{align*}
            	n\left(\rho\right) = n-r.
            \end{align*}
            Por tanto, $r(\rho) = r$, es decir $\{\lambda_{i},\ket{\varphi_{i}}\}^{r}_{i=1}$ forma un conjunto minimal para $\rho$.
            \item 
            \begin{itemize}
            	\item[$(a)$] Usando el hecho que $\rho$ es biyectivo cuando esta restringido en $\textup{supp}(\rho)$, se deduce para $\ket{\psi_{1}}\in \textup{supp}(\rho)$ que existe un único $\ket{\phi}\in \textup{supp}(\rho)$ tal que
            	\begin{align*}
            		\rho\ket{\phi} = c\ket{\psi_{1}},
            	\end{align*}  
                con lo cual, existe un único $\ket{\phi}\in \textup{supp}(\rho)$ tal que
                \begin{align*}
                	\ket{\phi} = c\rho^{-1}\ket{\psi_{1}}.
                \end{align*}
                Además,
                \begin{align*}
                	A_{p}\ket{\phi} &= \rho\ket{\phi} - p\ket{\psi_{1}}\braket{\psi_{1}|\phi}\\
                	                &= c\ket{\psi_{1}} - cp\bra{\psi_{1}}\rho^{-1}\ket{\psi_{1}}\ket{\psi_{1}}\\
                	                &= c\left(1- p\bra{\psi_{1}}\rho^{-1}\ket{\psi_{1}}\right)\ket{\psi_{1}},
                \end{align*}
                luego, la única forma en que $A_{p}\ket{\phi} = 0$ es cuando 
                \begin{align*}
                	p = \frac{1}{\bra{\psi_{1}}\rho^{-1}\ket{\psi_{1}}}.
                \end{align*} 
                Ahora, para probar que $p_{1} = p$, se observa que
                \begin{align*}
                	\rho = \sum_{i=1}^{r}\lambda_{i}\ket{\varphi_{i}}\bra{\varphi_{i}}\ \textup{y}\ \rho = \sum_{i=1}^{r}p_{i}\ket{\psi_{i}}\bra{\psi_{i}}.
                \end{align*}
                Por resultado visto en clases, existe un unitario tal que 
                \begin{align*}
                	\sqrt{p_{j}}\ket{\psi_{j}} = \sum_{i=1}^{r}u_{ij}\sqrt{\lambda_{i}}\ket{\varphi_{i}}\quad \forall j\in\{1,\dots,r\}.
                \end{align*}
                Usando que $\textup{span}\left(\ket{\varphi_{1}},\dots,\ket{\varphi_{r}}\right) = \textup{supp}(\rho)$, se llega a que 
                \begin{align*}
                	\ket{\psi_{1}} = \sum^{r}_{i=1} \frac{\braket{\psi_{1}|\varphi_{i}}}{||\varphi_{i}||^{2}}\ket{\varphi_{i}} = \sum^{r}_{i=1} \braket{\psi_{1}|\varphi_{i}}\ket{\varphi_{i}},
                \end{align*}
                de esta forma 
                \begin{align*}
                	\sqrt{p_{1}}\sum^{r}_{i=1}\braket{\psi_{1}|\varphi_{i}}\ket{\varphi_{i}} = \sum_{i=1}^{r}u_{i1}\sqrt{\lambda_{i}}\ket{\varphi_{i}},
                \end{align*}
                luego,
                \begin{align*}
                	\sqrt{p_{1}}\sum^{r}_{i=1}\braket{\psi_{1}|\varphi_{i}}\braket{\varphi_{k}|\varphi_{i}} = \sum_{i=1}^{r}u_{i1}\sqrt{\lambda_{i}}\braket{\varphi_{k}|\varphi_{i}}\quad\forall k\in\{1,\dots,k\},
                \end{align*}
                entonces
                \begin{align*}
                	\sqrt{p_{1}}\braket{\psi_{1}|\varphi_{k}} = u_{k1} \sqrt{\lambda_{k}}\quad\forall k\in\{1,\dots,k\},
                \end{align*}
                de aquí se desprende que 
                \begin{align*}
                	u_{k1} = \sqrt{\frac{p_{1}}{\lambda_{k}}}\braket{\psi_{1}|\varphi_{k}}\quad\forall k\in\{1,\dots,k\},
                \end{align*}
                sumando y multiplicando por el conjugado
                \begin{align*}
                	1 = \sum^{r}_{k=1}|u_{k1}|^{2} = \sum^{r}_{k=1}u_{k1}\bar{u}_{k1} = \sum^{r}_{k=1} \frac{p_{1}}{\lambda_{k}}\braket{\psi_{1}|\varphi_{k}}\braket{\varphi_{k}|\psi_{1}} = p_{1}\bra{\psi_{1}}\left(\sum_{k=1}^{r}\lambda^{-1}_{k}\ket{\varphi_{k}}\bra{\varphi_{k}}\right)\ket{\psi_{1}} = p_{1}\bra{\psi_{1}}\rho^{-1}\ket{\psi_{1}}.
                \end{align*}
                De esta forma
                \begin{align*}
                	p_{1} = \frac{1}{\bra{\psi_{1}}\rho^{-1}\ket{\psi_{1}}} = p.
                \end{align*}
                \item[$(b)$] En primer lugar, se puede ver que 
                \begin{align*}
                	A_{p}\geq 0 \iff \rho^{-1/2}A_{p}\rho^{-1/2}\geq 0.
                \end{align*}
                desarrollando la expresión $\rho^{-1/2}A_{p}\rho^{-1/2}$, se llega
                \begin{align*}
                	\rho^{-1/2}A_{p}\rho^{-1/2} &= \rho^{-1/2}\left(\rho - p \ket{\psi_{1}}\bra{\psi_{1}}\right)\rho^{-1/2}\\
                	&= \mathds{1} - p\rho^{-1/2}\ket{\psi_{1}}\bra{\psi_{1}}\rho^{-1/2}.
                \end{align*}
                dado $\ket{\xi}\in \textup{supp}(\rho)$
                \begin{align*}
                	\bra{\xi}\left(\mathds{1} - p\rho^{-1/2}\ket{\psi_{1}}\bra{\psi_{1}}\rho^{-1/2}\right)\ket{\xi} &= \braket{\xi|\xi} - p \bra{\xi}\rho^{-1/2}\ket{\psi_{1}}\bra{\psi_{1}}\rho^{-1/2}\ket{\xi}\\
                	&= ||\xi||^{2} - p |\bra{\xi}\rho^{-1/2}\ket{\psi_{1}}|^{2}\\
                	&\geq ||\xi||^{2} - p_{1} |\bra{\xi}\rho^{-1/2}\ket{\psi_{1}}|^{2}.
                \end{align*}
                Por Cauchy Schwarz
                \begin{align*}
                	|\bra{\xi}\rho^{-1/2}\ket{\psi_{1}}|^{2}\leq ||\xi||^{2}||\rho^{-1/2}\ket{\psi_{1}}||^{2} = ||\xi||^{2}\bra{\psi_{1}}\rho^{-1/2}\rho^{-1/2}\ket{\psi_{1}} = ||\xi||^{2}\frac{1}{p_{1}}
                \end{align*}
                De esta forma
                \begin{align*}
                	\bra{\xi}\left(\mathds{1} - p\rho^{-1/2}\ket{\psi_{1}}\bra{\psi_{1}}\rho^{-1/2}\right)\ket{\xi} \geq ||\xi||^{2} - p_{1} |\bra{\xi}\rho^{-1/2}\ket{\psi_{1}}|^{2} \geq ||\xi||^{2} - \frac{p_{1}}{p_{1}}||\xi||^{2} = 0.
                \end{align*}
                Por tanto, 
                \begin{align*}
                	A_{p}\geq 0.
                \end{align*}
                Para determinar el rango de $A_{p_{1}}$, se puede observar que
                \begin{align*}
                	A_{p_{1}} = \rho - p_{1}\ket{\psi_{1}}\bra{\psi_{1}} = \sum^{r}_{i=1}p_{i}\ket{\psi_{i}}\bra{\psi_{i}} - p_{1}\ket{\psi_{1}}\bra{\psi_{1}} = \sum^{r}_{i=2}p_{i}\ket{\psi_{i}}\bra{\psi_{i}}.
                \end{align*}
                Luego, el $r\left(A_{p_{1}}\right) = r-1$.
                \item[$(c)$] Dado $\ket{\psi_{1}}\in\textup{supp}(\rho)$, luego, se tiene que 
                \begin{align*}
                	\rho = p\ket{\psi_{1}}\bra{\psi_{1}} + A_{p}
                \end{align*}
                luego, existe un único $\ket{\phi}\in\textup{supp}(\rho)$ y $p\in]0,1]$ tal que
                \begin{align*}
                	A_{p}\ket{\phi} = 0 
                \end{align*} 
                donde 
                \begin{align*}
                	p = p_{1} = \bra{\phi_{1}}\rho^{-1}\ket{\psi_{1}}^{-1},\quad \ket{\phi} = c\rho^{-1}\ket{\psi_{1}},
                \end{align*}
                Además $r\left(A_{p_{1}}\right) = r-1$, por lo que, por teorema espectral 
                \begin{align*}
                	A_{p_{1}} = \sum_{i=1}^{r-1}\nu_{i}\ket{\eta_{i}}\bra{\eta_{i}}
                \end{align*}
                de esta forma 
                \begin{align*}
                	\rho = p_{1}\ket{\psi_{1}}\bra{\psi_{1}} + \sum_{i=1}^{r-1}\nu_{i}\ket{\eta_{i}}\bra{\eta_{i}}
                \end{align*}
                dado $r\left(\rho\right) =  r$, se tiene que $\{\ket{\psi_{1}},p_{1},\ket{\eta_{1}},\nu_{1},\dots,\ket{\eta_{r-1}},\nu_{r-1}\}$ es un conjunto minimal para $\rho$. Además, $\textup{Proba}(\psi_{1}) = p_{1}$.
            \end{itemize}
    	\end{enumerate}
    \end{proof}
    \begin{pro}
    	\textit{Purificación del estado de salida del canal cuántico.}\\
    	Sea $\rho_{S}$ un operador densidad sobre un espacio de Hilbert $\mathcal{H}_{S}$ y $\mathcal{M}:\mathcal{B}\left(\mathcal{H}_{S}\right)\longrightarrow\mathcal{B}\left(\mathcal{H}_{S}\right)$ un canal cuántico. Sea $\ket{\Psi_{SA}}$ una purificación de $\rho_{S}$ sobre $\mathcal{H}_{S}\otimes\mathcal{H}_{A}$. Sea $\mathcal{H}_{E}$ un espacio de Hilbert, $U_{SE}$ el unitario sobre $\mathcal{H}_{S}\otimes\mathcal{H}_{E}$ y $\ket{\epsilon_{0}}\in\mathcal{H}_{E}$ el estado puro en la dilatación de Stinespring de $\mathcal{M}$,
    	\begin{align*}
    		\mathcal{M}\left(\rho_{S}\right) = \underset{E}{\textup{tr}}\left(U_{SE}\rho_{S}\otimes\ket{\epsilon_{0}}\bra{\epsilon_{0}}U^{\dagger}_{SE}\right)
    	\end{align*}
        \begin{enumerate}
        	\item Muestre que 
        	\begin{align*}
        		\ket{\Psi^{\mathcal{M}}_{SAE}} = U_{SE}\otimes\mathds{1}_{A}\ket{\Psi_{SA}}\ket{\epsilon_{0}}
        	\end{align*}
            es una purificación de $\mathcal{M}\left(\rho_{S}\right)$ sobre $\mathcal{H}_{S}\otimes\mathcal{H}_{A}\otimes\mathcal{H}_{E}$.
            \item Sea $\rho_{S} = \sum_{k}\mu_{k}\ket{\varphi_{k}}\bra{\varphi_{k}}$ la descomposición espectral de $\rho_{S}$ y $\{A_{i}\}$ una familia de operadores de Kraus para $\mathcal{M}$. Muestre que 
            \begin{align*}
            	\ket{\Psi^{\mathcal{M}}_{SAE}} = \sum_{k}\sum_{i}\sqrt{\mu_{k}}A_{i}\ket{\varphi_{k}}\ket{\alpha_{k}}\ket{\epsilon_{i}}
            \end{align*}
            es una purificación de $\mathcal{M}\left(\rho_{S}\right)$ sobre $\mathcal{H}_{S}\otimes\mathcal{H}_{A}\otimes\mathcal{H}_{E}$, donde $\{\ket{\alpha_{k}}\}$ y $\ket{\epsilon_{i}}$ son familias ortonormales arbitrarias de $\mathcal{H}_{A}$ y $\mathcal{H}_{E}$, respectivamente. 
        \end{enumerate}
    \end{pro}
    \begin{proof}
    	\begin{enumerate}
    		\item Dado que $\ket{\Psi_{SA}}$ es un estado puro sobre $\mathcal{H}_{S}\otimes\mathcal{H}_{A}$ y que $\ket{\epsilon_{0}}$ es un estado puro sobre $\mathcal{H}_{E}$ se puede deducir que $\ket{\Psi^{\mathcal{M}}_{SAE}}$ es un estado puro. Luego de un cálculo directo
    		\begin{align*}
    			\underset{AE}{\textup{tr}}\left(\ket{\Psi^{\mathcal{M}}_{SAE}}\bra{\Psi^{\mathcal{M}}_{SAE}}\right) &= \underset{A}{\textup{tr}}\ \underset{E}{\textup{tr}}\left(U_{SE}\otimes\mathds{1}_{A}\ket{\Psi_{SA}}\ket{\epsilon_{0}} \bra{\epsilon_{0}}\bra{\Psi_{SA}}\mathds{1}_{A}\otimes U^{\dagger}_{SE}\right)\\
    			&= \underset{E}{\textup{tr}}\ \underset{A}{\textup{tr}}\left(U_{SE}\otimes\mathds{1}_{A}\ket{\Psi_{SA}}\ket{\epsilon_{0}} \bra{\epsilon_{0}}\bra{\Psi_{SA}}\mathds{1}_{A}\otimes U^{\dagger}_{SE}\right)\\
    			&= \underset{E}{\textup{tr}}\ \underset{A}{\textup{tr}}\left(U_{SE}\otimes\mathds{1}_{A}\ket{\Psi_{SA}}\bra{\Psi_{SA}}\otimes\ket{\epsilon_{0}} \bra{\epsilon_{0}}\mathds{1}_{A}\otimes U^{\dagger}_{SE}\right)\\
    			&= \underset{E}{\textup{tr}}\left( U_{SE} \ \underset{A}{\textup{tr}}\left( \ket{\Psi_{SA}}\bra{\Psi_{SA}}\otimes\ket{\epsilon_{0}} \bra{\epsilon_{0}}\right)U^{\dagger}_{SE} \right)\\
    			&= \underset{E}{\textup{tr}}\left( U_{SE} \ \underset{A}{\textup{tr}}\left( \ket{\Psi_{SA}}\bra{\Psi_{SA}}\otimes\ket{\epsilon_{0}} \bra{\epsilon_{0}}\right)U^{\dagger}_{SE} \right)\\
    			&= \underset{E}{\textup{tr}}\left( U_{SE} \ \underset{A}{\textup{tr}}\left( \ket{\Psi_{SA}}\bra{\Psi_{SA}}\right)\ket{\epsilon_{0}} \bra{\epsilon_{0}} U^{\dagger}_{SE} \right)\\
    			&= \underset{E}{\textup{tr}}\left( U_{SE} \rho_{S}\otimes\ket{\epsilon_{0}} \bra{\epsilon_{0}} U^{\dagger}_{SE} \right)\\
    			&= \mathcal{M}\left(\rho_{S}\right)
    		\end{align*}
    	\item De un cálculo directo
    	\begin{align*}
    		\underset{AE}{\textup{tr}}\left(\ket{\Psi^{\mathcal{M}}_{SAE}}\bra{\Psi^{\mathcal{M}}_{SAE}}\right) &= \underset{E}{\textup{tr}}\underset{A}{\textup{tr}}\left(\sum_{kijl}\sqrt{\mu_{k}\mu_{j}}A_{i}\ket{\varphi_{k}}\ket{\alpha_{k}}\ket{\epsilon_{i}}\bra{\epsilon_{l}}\bra{\alpha_{j}}\bra{\varphi_{j}}A^{\dagger}_{l}\right)\\
    		&= \underset{E}{\textup{tr}}\underset{A}{\textup{tr}}\left(\sum_{kijl}\sqrt{\mu_{k}\mu_{j}}A_{i}A^{\dagger}_{l}\ket{\varphi_{k}}\bra{\varphi_{j}}\ket{\alpha_{k}}\bra{\alpha_{j}}\ket{\epsilon_{i}}\bra{\epsilon_{l}}\right)\\
    		&= \sum_{kijl}\sqrt{\mu_{k}\mu_{j}}\braket{\alpha_{k}|\alpha_{j}}\underset{E}{\textup{tr}}\left(A_{i}A^{\dagger}_{l}\ket{\varphi_{k}}\bra{\varphi_{j}}\ket{\epsilon_{i}}\bra{\epsilon_{l}}\right)\\
    		&= \sum_{ijl}\sqrt{\mu_{j}\mu_{j}}\ \underset{E}{\textup{tr}}\left(A_{i}A^{\dagger}_{l}\ket{\varphi_{j}}\bra{\varphi_{j}}\ket{\epsilon_{i}}\bra{\epsilon_{l}}\right) \\
    		&= \sum_{ijl}\mu_{j} A_{i}A^{\dagger}_{l}\ket{\varphi_{j}}\bra{\varphi_{j}} \braket{\epsilon_{i}|\epsilon_{l}}\\
    		&= \rho_{S}\sum_{i}A_{i}A^{\dagger}_{i} \\
    		&= \mathcal{M}(\rho_{S}).
    	\end{align*}
    	\end{enumerate}
    \end{proof}
    \newpage
    \begin{pro}
    	\textit{Discriminación ambigua de 2 estados.}\\
    	Suponga que Alice prepara un sistema cuántico en uno de los dos estados $\rho_{1}$ o $\rho_{2}$ con
    	probabilidades $p_{1}$ y $p_{2} = 1-p_{1}$. Luego ella transmite el sistema a Bob. Con el fin de
    	determinar cual de los dos estados ha sido preparado por Alice, Bob realiza una medición
    	generalizada sobre el sistema, descrita por un POVM $\{M_{1},M_{2}\}$. El decide que el estado es $\rho_{1}$
    	si el resultado de la medición es “1” y es $\rho_{2}$ si el resultado de la medición es “2”. El propósito
    	de la discriminación ambigua de estados (o discriminación con error mínimo) es hallar la
    	medición óptima que minimiza la probabilidad de equivocación de Bob. Esta probabilidad
    	está dada por
    	\begin{align*}
    		P_{\textup{err}} = 1 - \sum_{i=1}^{2}p_{i}\textup{Proba}(i|i),
    	\end{align*}
        donde $\textup{Proba}(i|i) = \textup{tr}\left(M_{i}\rho_{i}\right)$ es la probabilidad del resultado $i$ sabiendo que el estado es $\rho_{i}$.
        \begin{enumerate}
        	\item Muestre que $P_{\textup{err}} = p_{1} - \textup{tr}\left(M_{1}\Lambda\right)$ con $\Lambda = p_{1}\rho_{1} - p_{2}\rho_{2}$.
        	\item Deduzca que la probabilidad de error mínima está dada por 
        	\begin{align*}
        		P_{\textup{err}} = \frac{1}{2}\left(1-\textup{tr}|\Lambda|\right)
        	\end{align*} 
            y que la medición óptima es $\{M^{\textup{op}}_{1} = \Pi_{+},M^{\textup{op}}_{2} = \mathds{1} - \Pi_{+}\}$, donde $\Pi_{+}$ es la proyección sobre el subespacio generado por los vectores propios de $\Lambda$ con valores propios positivos.
            \item Muestre que en el caso de dos estados puros equiprobables, $\rho_{i} = \ket{\psi_{i}}\bra{\psi_{i}}$ con $p_{i} = 1/2$, $i=1,2$, donde $\ket{\psi_{1}}$ y $\ket{\psi_{2}}$ no son ortogonales entre sí,
            \begin{align*}
            	P_{\textup{err}} = \frac{1}{2}\left(1-\sqrt{1-|\braket{\psi_{1}|\psi_{2}}|^{2}}\right).
            \end{align*} 
            Determine la base de medición optimal $\{\ket{\phi^{\textup{op}}_{1}},\ket{\phi^{\textup{op}}_{2}}\}$ en función de $\ket{\psi_{1}}$ y $\psi_{2}$ y represente gráficamente en el plano los vectores $\ket{\psi_{1}},\ket{\psi_{2}},\ket{\phi^{\textup{op}}_{1}}$ y $\ket{\phi^{\textup{op}}_{2}}$.
        \end{enumerate}
    \end{pro}
    \begin{proof}
    	\begin{enumerate}
    		\item Del enunciado se sabe
    		\begin{align*}
    			P_{\textup{err}} &= 1 - \sum_{i=1}^{2}p_{i}\textup{Proba}(i|i)\\
    			                 &= 1 - p_{1}\textup{tr}\left(M_{1}\rho_{1}\right) - p_{2}\textup{tr}\left(M_{2}\rho_{2}\right)\\
    			                 &= 1 - p_{1}\textup{tr}\left(M_{1}\rho_{1}\right) - p_{2}\textup{tr}\left(\left(\mathds{1} - M_{1}\right)\rho_{2}\right)\\
    			                 &= 1 - p_{1}\textup{tr}\left(M_{1}\rho_{1}\right) - p_{2}\textup{tr}\left(\rho_{2}\right) + p_{2}\textup{tr}\left(M_{1}\rho_{2}\right)\\
    			                 &=  1 - p_{1}\textup{tr}\left(M_{1}\rho_{1}\right) - p_{2} + p_{2}\textup{tr}\left(M_{1}\rho_{2}\right)\\
    			                 &= 1-p_{2}  - \textup{tr}\left(p_{1}M_{1}\rho_{1} - p_{2}M_{1}\rho_{2}\right)\\
    			                 &= p_{1}  - \textup{tr}\left(M_{1}\Lambda\right),
    		\end{align*}
    	    con $\Lambda := p_{1}\rho_{1} - p_{2}\rho_{2}$.
    	    \item En vista que $\Lambda$ no necesariamente es un operador positivo, considere la descomposición
    	    \begin{align*}
    	    	\Lambda = \Lambda_{+} - \Lambda_{-}
    	    \end{align*}
            donde 
            \begin{align*}
            	\Lambda_{\pm} = \frac{|\Lambda|\pm \Lambda}{2}\geq 0.
            \end{align*}
            Luego, se tiene el siguiente problema de minimización
            \begin{align*}
            	\min_{M_{1}} \ &p_{1} - \textup{tr}\left(M_{1}\Lambda\right)\\
            	\textup{s.a.}\ &M_{1} + M_{2} = \mathds{1},\\
            	&M_{1},M_{2}\geq 0.
            \end{align*}
            el cual es equivalente al problema de maximización
            \begin{align*}
            	\max_{M_{1}} \ &\textup{tr}\left(M_{1}\Lambda\right)\\
            	\textup{s.a.}\ &M_{1} + M_{2} = \mathds{1},\\
            	&M_{1},M_{2}\geq 0.
            \end{align*} 
            Desarrollando la función objetivo
            \begin{align*}
            	\textup{tr}\left(M_{1}\Lambda\right) = \textup{tr}\left(M_{1}\Lambda_{+}\right) - \textup{tr}\left(M_{1}\Lambda_{-}\right).
            \end{align*}
            En vista que $M_{1}\Lambda_{+}$ y $M_{1}\Lambda_{-}$ son operadores no negativos se deduce que
            \begin{align*}
            	\textup{tr}\left(M_{1}\Lambda_{+}\right) - \textup{tr}\left(M_{1}\Lambda_{-}\right) \geq 0.
            \end{align*}
            Por lo que el máximo se encuentra cuando $\textup{tr}\left(M_{1}\Lambda_{-}\right) = 0$ y cuando $\textup{tr}\left(M_{1}\Lambda_{+}\right) = \textup{tr}\left(\Lambda_{+}\right)$. Lo anterior ocurre cuando $M_{1} = \Pi_{+}$, donde $\Pi_{+}$ es la proyección sobre el subespacio generado por los vectores propios de $\Lambda$ con valores propios positivos. Además, dado que
            \begin{align*}
            	M_{1} + M_{2} = \mathds{1},
            \end{align*}
            se deduce que 
            \begin{align*}
            	M_{2} = \mathds{1} - \Pi_{+}.
            \end{align*}
            De esta forma la medición optima es $\{M^{\textup{op}}_{1} = \Pi_{+},M^{\textup{op}}_{2} = \mathds{1} - \Pi_{+}\}$, teniendo en cuenta esto, se deduce que 
            \begin{align*}
            	P_{\textup{err}} &= p_{1} - \tr{M^{\textup{op}}_{1}\Lambda}\\
            	                 &= p_{1} - \tr{\Lambda_{+}}\\
            	                 &= p_{1} - \frac{1}{2}\tr{|\Lambda| + \Lambda}\\
            	                 &= p_{1} - \frac{1}{2}\textup{tr}|\Lambda| - \frac{1}{2}\tr{\Lambda}\\
            	                 &= p_{1} - \frac{1}{2}\textup{tr}|\Lambda| - \frac{1}{2}\left(p_{1}\tr{\rho_{1}} + p_{2}\tr{\rho_{2}}\right)\\
            	                 &=  p_{1} - \frac{1}{2}\textup{tr}|\Lambda| - \frac{p_{1}}{2} + \frac{p_{2}}{2}\\
            	                 &= \frac{p_{1}}{2} + \frac{p_{2}}{2} - \frac{1}{2}\textup{tr}|\Lambda|\\
            	                 &= \frac{1}{2}\big(1-\textup{tr}|\Lambda|\big)
            \end{align*}
            \item Del \textbf{Problema 6.2} se sabe que 
             \begin{align*}
             	P_{\textup{err}} = \frac{1}{2}\left(1-\textup{tr}|\Lambda|\right).
             \end{align*}
             Por lo que basta con demostrar que $\textup{tr}|\Lambda| = sqrt{1 - |\braket{\psi_{1}|\psi_{2}}|^{2}}$. Para esto,
             sea $\mathcal{H}$ un espacio de Hilbert, en el cual $\ket{\psi_{1}},\ket{\psi_{2}}\in \mathcal{H}$ y ademas se define $S := \textup{span}\left(\ket{\psi_{1}}\right)$, por teorema de descomposición ortogonal
            \begin{align*}
            	\mathcal{H} = S\oplus S^{\perp}.
            \end{align*}
            De aquí se infiere que existen constantes $\alpha,\beta\in \mathbb{C}$ tales que 
            \begin{align*}
            	\ket{\psi_{2}} = \alpha \ket{\psi_{1}} + \beta \ket{\xi}
            \end{align*}
            donde $\ket{\xi}\in S^{\perp}$. Mediante un simple calculo simple, se deduce que 
            \begin{align*}
            	1  = |\alpha|^{2} + |\beta|^{2}
            \end{align*}
            y también se puede deducir de calcular $\braket{\psi_{2}|\psi_{2}}$ que 
            \begin{align*}
            	\alpha = \braket{\psi_{1}|\psi_{2}}
            \end{align*}
            Por lo que 
            \begin{align*}
            	|\beta|^{2} = 1 - |\braket{\psi_{1}|\psi_{2}}|^{2}.
            \end{align*}
            Por otro lado
            \begin{align*}
            	\rho_{2} &= \ket{\psi_{2}}\bra{\psi_{2}}\\
            	         &= \left(\alpha \ket{\psi_{1}} + \beta \ket{\xi}\right)\left(\bar{\alpha} \bra{\psi_{1}} + \bar{\beta} \bra{\xi}\right)\\
            	         &= |\alpha|^{2}\rho_{1} + \alpha\bar{\beta}\ket{\psi_{1}}\bra{\xi} + \beta\bar{\alpha}\ket{\xi}\bra{\psi_{1}} + |\beta|^{2}\ket{\xi}\bra{\xi}.
            \end{align*}
            Luego
            \begin{align*}
            	\Lambda &= \frac{1}{2}\left(\rho_{1} - |\alpha|^{2}\rho_{1} - \alpha\bar{\beta}\ket{\psi_{1}}\bra{\xi} - \beta\bar{\alpha}\ket{\xi}\bra{\psi_{1}} - |\beta|^{2}\ket{\xi}\bra{\xi}\right)\\
            	&=  \frac{1}{2}\left( |\beta|^{2}\rho_{1} - \alpha\bar{\beta}\ket{\psi_{1}}\bra{\xi} - \beta\bar{\alpha}\ket{\xi}\bra{\psi_{1}} - |\beta|^{2}\ket{\xi}\bra{\xi}\right)\\
            	&= \frac{1}{2}\left(|\beta|^{2} \rho_{1} - \left(\alpha\bar{\beta}\ket{\psi_{1}}\bra{\xi} + \beta\bar{\alpha}\ket{\xi}\bra{\psi_{1}} + |\beta|^{2}\ket{\xi}\bra{\xi}\right)\right),
            \end{align*}
            Por lo que 
            \begin{align*}
            	\Lambda\Lambda^{\dagger} &= \frac{1}{4}\left(|\beta|^{2}\ket{\psi_{1}}\bra{\psi_{1}} - \alpha\bar{\beta}\ket{\psi_{1}}\bra{\xi} - \beta\bar{\alpha}\ket{\xi}\bra{\psi_{1}} - |\beta|^{2}\ket{\xi}\bra{\xi}\right)\left(|\beta|^{2}\ket{\psi_{1}}\bra{\psi_{1}} - \overline{\alpha\bar{\beta}}\ket{\xi}\bra{\psi_{1}} - \overline{\beta\bar{\alpha}}\ket{\psi_{1}}\bra{\xi} - |\beta|^{2}\ket{\xi}\bra{\xi}\right)\\
            	&= \frac{1}{4}\left(|\beta|^{4}\ket{\psi_{1}}\bra{\psi_{1}} - \beta\bar{\alpha}|\beta|^{2}\ket{\xi}\bra{\psi_{1}} + |\alpha\bar{\beta}|^{2}\ket{\psi_{1}}\bra{\psi_{1}} + |\beta|^{2}\overline{\alpha\bar{\beta}}\ket{\xi}\bra{\psi_{1}} - \overline{\beta\bar{\alpha}}|\beta|^{2}\ket{\psi_{1}}\bra{\xi} + |\beta\bar{\alpha}|^{2}\ket{\xi}\bra{\xi}\right.\\
            	&+ \left.\alpha\bar{\beta}|\beta|^{2}\ket{\psi_{1}}\bra{\xi} + |\beta|^{4}\ket{\xi}\bra{\xi}\right)\\
            	&= \frac{1}{4}\left(|\beta|^{4} + |\alpha|^{2}|\beta|^{2}\right)\ket{\psi_{1}}\bra{\psi_{1}} + \frac{1}{4}\left(|\beta|^{4} + |\alpha|^{2}|\beta|^{2}\right)\ket{\xi_{1}}\bra{\xi_{1}}\\
            	&= \frac{|\beta|^{2}}{4}\left(\ket{\psi_{1}}\bra{\psi_{1}} + \ket{\xi}\bra{\xi}\right)\\
            	&= \frac{|\beta|^{2}}{4}\mathds{1}.
            \end{align*}
            Así,
            \begin{align*}
            	\textup{tr}|\Lambda| = \textup{tr}\sqrt{\Lambda\Lambda^{\dagger}} = |\beta| = \sqrt{1 - |\braket{\psi_{1}|\psi_{2}}|^{2}}.
            \end{align*}
            Por tanto
            \begin{align*}
            		P_{\textup{err}} = \frac{1}{2}\left(1-\sqrt{1-|\braket{\psi_{1}|\psi_{2}}|^{2}}\right).
            \end{align*}
            Del álgebra lineal, se deduce que la base de medición optimal es $\{\ket{\phi^{\textup{op}}_{1}},\ket{\phi^{\textup{op}}_{2}}\}$ donde, $\ket{\phi^{\textup{op}}_{1}} = \ket{\psi_{1}}\in S$ y $\ket{\phi^{\textup{op}}_{2}} = \ket{\psi_{1}} - \ket{\psi_{2}}\in S^{\perp}.$ Gráficamente
            \begin{figure}[h]
            	 \begin{center}
            	 	\begin{picture}(20,60)
            	 		\put(0,0){\vector(1,0){50}}
            	 		\put(0,0){\vector(1,1){40}}
            	 		\put(0,0){\vector(-1,2){20}}
            	 		\put(-62,40){$\ket{\phi^{\textup{op}}_{2}}$}
            	 		\put(-30,44){$\ket{\psi_{2}}$}
            	 		\put(-5,55){$\ket{1}$}
            	 		\put(55,0){$\ket{0}$}
            	 		\put(0,0){\vector(0,1){50}}
            	 		\put(0,0){\vector(-1,1){40}}\put(40,40){$\ket{\phi^{\textup{op}}_{1}}$}
            	 	\end{picture}
            	 \end{center}
            \end{figure}
    	\end{enumerate}
    \end{proof}
    \newpage
    \begin{pro}
    	\textit{Discriminación no ambigua de 2 estados puros.}\\
    	Una estrategia alternativa a la contemplada en el problema anterior para discriminar
    	estados consiste en tratar de identificar con certeza el estado preparado por Alice a costa de
    	tener un resultado inconcluso. Más precisamente, Bob hace una medición con $3$ resultados,
    	descrita por un POVM $\{M_{j}\}^{2}_{j=0}$. Si el obtiene el resultado “1”, el estado es con certeza $\rho_{}1$,
    	si el obtiene “2” el estado es con certeza $\rho_{2}$, y si el obtiene “0” no se sabe. Queremos hallar
    	una medición óptima que minimiza la probabilidad $Q_{0}$ del resultado inconcluso “0”,
    	\begin{align*}
    		Q_{0} = \sum^{2}_{i=1}p_{i}P_{0|i},
    	\end{align*}
    	donde $P_{0|i} = \tr{M_{0}\rho_{i}}$ es la probabilidad del resultado “0” sabiendo que el estado es $\rho_{i}$. La condición de no ambigüedad se escribe $P_{j|i} = \tr{M_{j}\rho_{j}} = 0$ si $i\neq j$, $i,j=1,2.$ En este problema suponemos que Alice prepara estados puros $\rho_{i} = \ket{\psi_{i}}\bra{\psi_{i}}$ con probabilidad $p_{i}$, $i=1,2.$ Sin perdida de la generalidad, podemos suponer que el espacio de Hilbert tiene dimensión $2$, esto es, $\mathcal{H} = \textup{span}\{\ket{\psi_{1}},\ket{\psi_{2}}\}$,
    	\begin{enumerate}
    		\item Muestre que la condición de no ambigüedad implica que 
    		\begin{align*}
    			M_{i} = P_{i|i}\ket{\tilde{\psi}^{*}_{i}}\bra{\tilde{\psi}^{*}_{i}}, \quad i= 1,2,
    		\end{align*} 
    	    donde $\{\ket{\tilde{\psi}^{*}_{i}}\}^{2}_{i=1}$ es la base dual de $\{\ket{\psi_{i}}\}^{2}_{i=1}$ y $P_{i|i} = 1-P_{0|i}>0$ es la probabilidad del resultado $i$ sabiendo que el estado es $\ket{\psi_{i}}$. Deduzca que para que la discriminación no ambigua sea posible, $P_{1|1}$ y $P_{2|2}$ deben ser tales que 
    	    \begin{align*}
    	    	\sum_{i=1}^{2}P_{i|i}\ket{\tilde{\psi}^{*}_{i}}\bra{\tilde{\psi}^{*}_{i}}\leq \mathds{1}.
    	    \end{align*}
            ¿El POVM $\{M_{j}\}^{2}_{j=1}$ puede ser asociado a una medición de von Neumann?
            \item En lo que sigue consideramos una medición generalizada $\{A_{j}\}^{2}_{j=0}$ asociada a $\{M_{j}\}^{2}_{j=0}$. Denotamos por $U_{SA}$, $\{\ket{\alpha_{j}}\}^{2}_{j=0}$ y $\ket{0}$ el operador unitario sobre $\mathcal{H}_{SA}$, la base ortonormal de medición sobre el ancilla $A$, y el estado inicial de $A$ asociado a $\{A_{j}\}^{2}_{j=0}$ en el teorema de extensión de Neumark. Sea
            \begin{align*}
            	\ket{\Psi_{i}^{SA}} = U_{SA}\ket{\psi_{i}}\ket{0},\quad \braket{\alpha_{j}|\Psi_{i}^{SA}} = C_{ji}\ket{\varphi_{j|i}},
            \end{align*}
            donde $\ket{\varphi_{j|i}}$ es un vector normalizado del espacio de Hilbert $\mathcal{H}_{S}$ del sistema y $C_{ji}\geq0$.
            \begin{itemize}
            	\item[$(a)$] Muestre que $C^{2}_{ji} = P_{j|i}$ y que $\ket{\varphi_{j|i}}$ es el estado del sistema después de la medición con resultado $j$ sabiendo que el estado inicial es $\ket{\psi_{i}}$.
            	\item[$(b)$] Usando la condición de no ambigüedad $P_{1|2} = P_{2|1} = 0$, muestre que 
            	\begin{align*}
            		\braket{\Psi_{1}^{SA}|\Psi_{2}^{SA}} = \sqrt{P_{0|1}P_{0|2}}\braket{\varphi_{0|1}|\varphi_{0|2}}
            	\end{align*} 
            	\item[$(c)$] Deduzca que 
            	\begin{align*}
            		P_{0|1}P_{0|2}\geq\cos^{2}\theta\quad\textup{con}\quad \cos\theta = |\braket{\psi_{1}|\psi_{2}}|,
            	\end{align*}
                y que la desigualdad es una igualdad si y sólo si $\ket{\varphi_{0|2}}$ coincide con $\ket{\varphi_{0|1}}$ módulo un factor de fase.
            \end{itemize}
            \item Muestre que si $\sqrt{p_{2}/p_{1}}\cos\theta \leq 1$ y $\sqrt{p_{1}/p_{2}}\cos\theta \leq 1$, luego la probabilidad mínima del resultado inconclusivo está dada por 
            \begin{align*}
                Q^{\textup{op}}_{0} = \min_{P_{0|1}\in[\cos^{2}\theta,1]}\left\{p_{1}P_{0|1} + p_{2}\frac{\cos^{2}\theta}{P_{0|1}}\right\} = 2\sqrt{p_{1}p_{2}}|\braket{\psi_{1}|\psi_{2}}|.
            \end{align*}
            \item Similarmente, halle el mínimo de $Q_{0}$ en los casos  $\sqrt{p_{2}/p_{1}}\cos\theta \geq 1$ y $\sqrt{p_{1}/p_{2}}\cos\theta \geq 1$.
    	\end{enumerate}
    \end{pro}
    \newpage
    \begin{proof}
    	\begin{enumerate}
    		\item Dado que $\{\ket{\tilde{\psi}^{*}_{i}}\}^{2}_{i=1}$ es una base, los $M_{i}$ se pueden obtener mediante 
    		\begin{align*}
    			M_{i} = c_{1i}\ket{\tilde{\psi}^{*}_{1}}\bra{\tilde{\psi}^{*}_{1}} +  c_{4i}\ket{\tilde{\psi}^{*}_{2}}\bra{\tilde{\psi}^{*}_{2}}, \quad \forall i\in\{1,2\}
    		\end{align*}
    	    Ademas,
    	    \begin{align*}
    	    	P_{1|1} &= \tr{M_{1}\rho_{1}}\\
    	    	&= c_{11}\tr{\ket{\tilde{\psi}^{*}_{1}}\braket{\tilde{\psi}^{*}_{1}|\psi_{1}}\bra{\psi_{1}}}c_{41}\tr{\ket{\tilde{\psi}^{*}_{2}}\braket{\tilde{\psi}^{*}_{2}|\psi_{1}}\bra{\psi_{1}}}\\
    	    	&= c_{11}\tr{\ket{\tilde{\psi}^{*}_{1}}\bra{\psi_{1}}} \\
    	    	&= c_{11} \\
    	    	P_{1|2} &= \tr{M_{1}\rho_{2}}\\
    	    	&= c_{11}\tr{\ket{\tilde{\psi}^{*}_{1}}\braket{\tilde{\psi}^{*}_{1}|\psi_{2}}\bra{\psi_{2}}}+c_{21}\tr{\ket{\tilde{\psi}^{*}_{2}}\braket{\tilde{\psi}^{*}_{2}|\psi_{2}}\bra{\psi_{2}}}\\
    	    	&= c_{41}\\
    	    	P_{2|1} &= \tr{M_{2}\rho_{1}}\\
    	    	&= c_{12}\tr{\ket{\tilde{\psi}^{*}_{1}}\braket{\tilde{\psi}^{*}_{1}|\psi_{1}}\bra{\psi_{1}}}+c_{22}\tr{\ket{\tilde{\psi}^{*}_{2}}\braket{\tilde{\psi}^{*}_{2}|\psi_{1}}\bra{\psi_{1}}}\\
    	    	&= c_{12}\\
    	    	P_{2|2} &= \tr{M_{2}\rho_{2}}\\
    	    	&= c_{12}\tr{\ket{\tilde{\psi}^{*}_{1}}\braket{\tilde{\psi}^{*}_{1}|\psi_{2}}\bra{\psi_{2}}}+c_{22}\tr{\ket{\tilde{\psi}^{*}_{2}}\braket{\tilde{\psi}^{*}_{2}|\psi_{2}}\bra{\psi_{2}}}\\
    	    	&= c_{22}.
    	    \end{align*} 
            Por la condición de no ambigüedad, $c_{12} = c_{21} = 0$. Y por tanto 
            \begin{align*}
            	M_{i} = P_{i|i}\ket{\tilde{\psi}^{*}_{i}}\bra{\tilde{\psi}^{*}_{i}}\quad \forall i\in\{1,2\}.
            \end{align*} 
            Por otro lado, dado que $\{M_{i}\}^{2}_{i=0}$ es un POVM
            \begin{align*}
            	M_{0} + M_{1} + M_{2} = \mathds{1}
            \end{align*}
            por lo que necesariamente 
            \begin{align*}
            	M_{1} + M_{2} \leq \mathds{1}.
            \end{align*}
            Además, el POVM $\{M_{j}\}^{2}_{j=1}$ puede ser asociado a una medición de von Neumann siempre y cuando $\{\ket{\tilde{\psi}^{*}_{i}}\}^{2}_{i=1}$ sea una base ortonormal.
            \item 
            \begin{itemize}
            	\item[$(a)$] Por el teorema de extensión de Neumark
            	\begin{align*}
            		A_{j}\rho_{i}A^{\dagger}_{j} &= \underset{A}{\textup{tr}}\left(\mathds{1}_{S}\otimes\ket{\alpha_{j}}\bra{\alpha_{j}}U_{SA}\rho_{i}\otimes\ket{0}\bra{0}U_{SA}^{\dagger}\mathds{1}_{S}\otimes\ket{\alpha_{j}}\bra{\alpha_{j}}\right)\\ &=  \underset{A}{\textup{tr}}\left(\mathds{1}_{S}\otimes\ket{\alpha_{j}}\braket{\alpha_{j}|\Psi_{i}^{SA}}\bra{\Psi_{i}^{SA}}\mathds{1}_{S}\otimes\ket{\alpha_{j}}\bra{\alpha_{j}}\right)\\ &= \braket{\alpha_{j}|\Psi_{i}^{SA}}\braket{\Psi_{i}^{SA}|\alpha_{j}},
            	\end{align*}
                luego,
                \begin{align*}
                	P_{j|i} = \tr{\braket{\alpha_{j}|\Psi_{i}^{SA}}\braket{\Psi_{i}^{SA}|\alpha_{j}}} = C^{2}_{ji}\tr{\ket{\varphi_{j|i}}\bra{\varphi_{j|i}}} = C^{2}_{ji}.
                \end{align*}
                Además,
                \begin{align*}
                	\rho_{j|i}\ket{\varphi_{j|i}} &= \frac{A_{j}\rho_{i}A^{\dagger}_{j}}{P_{i|j}}\\ &= \frac{\braket{\alpha_{j}|\Psi_{i}^{SA}}\braket{\Psi_{i}^{SA}|\alpha_{j}}}{P_{i|j}}\ket{\varphi_{j|i}}\\ &= \frac{\braket{\alpha_{j}|\Psi_{i}^{SA}}}{P_{i|j}}C_{ji}\braket{\varphi_{j|i}|\varphi_{j|i}}\\  &= \frac{\braket{\alpha_{j}|\Psi_{i}^{SA}}}{P_{i|j}}\sqrt{P_{i|j}}\\ &= \frac{\sqrt{P_{j|i}}}{P_{j|i}}\sqrt{P_{j|i}}\ket{\varphi_{j|i}}\\ &= \ket{\varphi_{j|i}}
                \end{align*}
                \item[$(b)$] De un cálculo directo
                \begin{align*}
                	\braket{\Psi^{SA}_{1}|\Psi^{SA}_{2}} &= \bra{\Psi^{SA}_{1}}\mathds{1}\ket{\Psi^{SA}_{2}}\\ &= \sum_{j=0}^{2}\braket{\Psi^{SA}_{1}|\alpha_{j}}\braket{\alpha_{j}|\Psi^{SA}_{2}}\\ &= \sum_{j=0}^{2}C_{j1}\bra{\varphi_{j|1}}C_{j2}\ket{\varphi_{j|2}}\\ &= \sum_{j=0}^{2}C_{j2}C_{j1}\braket{\varphi_{j|1}|\varphi_{j|2}}\\ &= \sum_{j=0}^{2}\sqrt{P_{j|2}P_{j|1}}\braket{\varphi_{j|1}|\varphi_{j|2}}\\
                	&= \sqrt{P_{0|2}P_{0|1}}\braket{\varphi_{0|1}|\varphi_{0|2}}
                \end{align*}
                \item[$(c)$] Usando lo mostrado en \textbf{Problema 7.2.b} se deduce
                \begin{align*}
                	|\braket{\Psi^{SA}_{1}|\Psi^{SA}_{2}}|^{2} &= \braket{\Psi^{SA}_{1}|\Psi^{SA}_{2}}\braket{\Psi^{SA}_{2}|\Psi^{SA}_{1}}\\ &= \sqrt{P_{0|2}P_{0|1}}\braket{\varphi_{0|1}|\varphi_{0|2}}\sqrt{P_{0|1}P_{0|2}}\braket{\varphi_{0|1}|\varphi_{0|2}}\\ &= P_{0|1}P_{0|2}|\braket{\varphi_{0|1}|\varphi_{0|2}}|^{2}
                \end{align*}
                De aquí
                \begin{align}\label{eq:wea}
                	P_{0|1}P_{0|2} = \frac{|\braket{\Psi^{SA}_{1}|\Psi^{SA}_{2}}|^{2}}{|\braket{\varphi_{0|1}|\varphi_{0|2}}|^{2}} = \frac{|\bra{0}\bra{\psi_{1}}U^{\dagger}_{SA}U_{SA}\ket{\psi_{2}}\ket{0}|^{2}}{|\braket{\varphi_{0|1}|\varphi_{0|2}}|^{2}} = \frac{|\braket{\psi_{1}|\psi_{2}}|^{2}}{|\braket{\varphi_{0|1}|\varphi_{0|2}}|^{2}} = \frac{\cos^{2}\theta}{|\braket{\varphi_{0|1}|\varphi_{0|2}}|^{2}},
                \end{align}
                De esta identidad, si $\ket{\varphi_{0|1}}$ coincide con $\ket{\varphi_{0|2}}$ modulo factor de fase, se obtiene
                \begin{align*}
                	P_{0|1}P_{0|2} = \frac{\cos^{2}\theta}{|e^{-i\phi}\braket{\varphi_{0|1}|\varphi_{0|1}}|^{2}} = \cos^{2}\theta.
                \end{align*}
                Recíprocamente, si 
                \begin{align*}
                	P_{0|1}P_{0|2} = \cos^{2}\theta,
                \end{align*}
                entonces 
                \begin{align*}
                	\cos^{2}\theta = \frac{\cos^{2}\theta}{|\braket{\varphi_{0|1}|\varphi_{0|2}}|^{2}} \iff \cos^{2}\theta\left(1-\frac{1}{|\braket{\varphi_{0|1}|\varphi_{0|2}}|^{2}}\right) = 0 \iff \psi_{1}\perp\psi_{2} \ \vee \ |\braket{\varphi_{0|1}|\varphi_{0|1}}|^{2} = 1
                \end{align*}
            
            Como en este no se asume ortogonalidad de entre $\psi_{1}$ y $\psi_{2}$, se tiene que 
            \begin{align*}
            	|\braket{\varphi_{0|1}|\varphi_{0|1}}|^{2} = 1
            \end{align*} 
            lo cual implica que $\ket{\varphi_{0|1}}$ coincide con $\ket{\varphi_{0|2}}$ modulo un factor de fase. Por otro lado, para probar la desigualdad notar que
            \begin{align*}
            	|\braket{\varphi_{0|1}|\varphi_{0|1}}|^{2} \leq ||\varphi_{0|1}||^{2}||\varphi_{0|2}||^{2} = 1 \Longrightarrow 1 \leq \frac{1}{|\braket{\varphi_{0|1}|\varphi_{0|1}}|^{2}}
            \end{align*}
            retomando \eqref{eq:wea} se deduce que
            \begin{align*}
            	P_{0|1}|P_{0|2} = \frac{\cos^{2}\theta}{|\braket{\varphi_{0|1}|\varphi_{0|1}}|^{2}}\geq \cos^{2}\theta 
            \end{align*}
        \end{itemize}
        \newpage
        \item Suponga que $\sqrt{p_{2}/p_{1}}\cos\theta \leq 1$ y $\sqrt{p_{1}/p_{2}}\cos\theta \leq 1$, luego, por definición y por lo mostrado en \textbf{Problema 7.2.c}
        \begin{align*}
        	Q_{0} = p_{1}P_{0|1} + p_{2}P_{0|2}\geq p_{1}P_{0|1} + p_{2}\frac{\cos^{2}\theta}{P_{0|1}},
        \end{align*}
        de esta forma, encontrar $P_{0|1}$ y $P_{0|2}$ tal que minimiza $Q_{0}$ se puede reducir a encontrar $P_{0|1}$ tal que minimiza $Q_{0}$, además, dado que 
        \begin{align*}
        	P_{0|1}P_{0|2}\geq \cos^{2}\theta \quad P_{0|1},P_{0|2}\leq 1,
        \end{align*}
        se deduce que $P_{0|1}\in [\cos^{2}\theta,1]$, es decir
        \begin{align*}
        	Q^{\textup{op}}_{0} = \min_{P_{0|1}\in[\cos^{2}\theta,1]}\left\{p_{1}P_{0|1} + p_{2}\frac{\cos^{2}\theta}{P_{0|1}}\right\}.
        \end{align*}
        Dado que la función objetivo $f:[\cos^{2}\theta,1]\longrightarrow\mathbb{R}$ es continua en un intervalo cerrado y acotado, se deduce que tiene mínimo. Ahora derivando respecto a $P_{0|1}$ e igualando a cero
        \begin{align*}
        	0 = \frac{\textup{d}f}{\textup{d}P_{0|1}} = p_{1} - p_{2}\cos^{2}\theta\frac{1}{P^{2}_{0|1}} \iff P_{0|1} = \pm\sqrt{\frac{p_{2}\cos^{2}\theta}{p_{1}}}.
        \end{align*}
        Dado que $P_{0|1}\in [\cos^{2}\theta,1]$, y que $\sqrt{p_{2}/p_{1}}\cos\theta \leq 1$ entonces el único punto critico de interés es
        \begin{align*}
        	P_{0|1} = \sqrt{\frac{p_{2}}{p_{1}}}\cos\theta
        \end{align*}
        luego, analizando la concavidad de $f$
        \begin{align*}
        	\frac{\textup{d}^{2}f}{\textup{d}P_{0|1}^{2}} = \frac{\textup{d}}{\textup{d}P_{0|1}}\left( p_{1} - p_{2}\cos^{2}\theta\frac{1}{P^{2}_{0|1}}\right) = 2p_{2}\cos^{2}\theta\frac{1}{P^{3}_{0|1}}>0 \quad \textup{en} [\cos^{2}\theta,1]
        \end{align*}
        Por tanto, $P_{0|1} = \sqrt{\frac{p_{2}\cos^{2}\theta}{p_{1}}}$ es un mínimo local y así
        \begin{align*}
        	Q^{\textup{op}}_{0} = p_{1}\sqrt{\frac{p_{2}}{p_{1}}}\cos\theta + p_{2}\sqrt{\frac{p_{1}}{p_{2}}}\cos\theta = 2\sqrt{p_{1}p_{1}}\cos\theta = 2\sqrt{p_{1}p_{1}}\  |\braket{\psi_{1}|\psi_{2}}|
        \end{align*}
        \item Suponga $\sqrt{p_{2}/p_{1}}\cos\theta \geq 1$, luego, $\cos^{2}\theta\geq \frac{p_{1}}{p_{2}}$. Por definición y por lo demostrado en el \textbf{Problema 7.3.c}
        \begin{align*}
        	Q_{0} = p_{1}P_{0|1} + p_{2}P_{0|2} \geq  p_{1}P_{0|1} + p_{2}\frac{\cos^{2}\theta}{P_{0|1}}\geq p_{1}\left(P_{0|1} + \frac{1}{P_{0|1}}\right)
        \end{align*}
        Procediendo de forma análoga al \textbf{Problema 7.3}, buscamos minimizar la función $f:[\cos^{2}\theta,1]\longrightarrow\mathbb{R}$ definida por 
        \begin{align*}
        	f\left(P_{0|1}\right) = p_{1}\left(P_{0|1} + \frac{1}{P_{0|1}}\right)
        \end{align*}
        derivando e igualando a cero
        \begin{align*}
        	0 = \frac{\textup{d}f}{\textup{d}P_{0|1}} = p_{1} - \frac{p_{1}}{P^{2}_{0|1}} \Longrightarrow P_{0|1} = \pm 1.
        \end{align*}
        Dado que $P_{0|1}\in [\cos^{2}\theta,1]$, se deduce que el único punto critico de interés es
        \begin{align*}
        	P_{0|1} = 1.
        \end{align*}
        Notar que si $P_{0|1}\leq 1$, se tiene que 
        \begin{align*}
        	\frac{\textup{d}f}{\textup{d}P_{0|1}} <0
        \end{align*}
        Por lo que la función es decreciente en $[\cos^{2}\theta,1]$ y por tanto, como $f$ es continua y monótona decreciente en un intervalo cerrado y acotado, se tiene que la función alcanza su mínimo en el punto $P_{0|1} = 1$. Por tanto
        \begin{align*}
        	Q^{\textup{op}}_{0} = 2p_{1}
        \end{align*}
        Por otro lado, si $\sqrt{p_{1}/p_{2}}\cos\theta \geq 1$, luego, $\cos^{2}\theta\geq \frac{p_{2}}{p_{1}}$. Por definición y por lo demostrado en el \textbf{Problema 7.3.c}
        \begin{align*}
        	Q_{0} = p_{1}P_{0|1} + p_{2}P_{0|2} \geq  p_{2}\frac{\cos^{2}\theta}{P_{0|2}} + p_{2}P_{0|2}\geq p_{1}\left(\frac{1}{P_{0|2}} + P_{0|2}\right)
        \end{align*}
        Luego procediendo de forma análoga al caso anterior, se deduce que
        \begin{align*}
        	Q^{\textup{op}}_{0} = 2p_{2}.
        \end{align*} 
    	\end{enumerate}
    \end{proof}
\end{document}